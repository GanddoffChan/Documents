\documentclass[a4paper]{article}
\usepackage[]{geometry}
\title{Tone Cluster Composition (draft 2): \textbf{Infree}}
\author{Gordon Chan}
\begin{document}
\maketitle
\section{Introduction}

The title of \textbf{Infree} describes the state of not being able to be free.
In this case, it describes the notes and voices in the piece being somewhat
strangled to one another, failed to separate and hence not free.
It also refers to the effect of tone clusters being rigid and not
scattered. Certain dynamic changes and rhythms are used to emphasise
the quality of the clusters.

\section{Harmony}

In the introductory motif,
a E minor tonality is suggested by the flute and violin.
That somewhat stable tonality is immediately destroyed at bar 2
when the tone clusters are introduced.
The following bar are tone clusters made up of major seconds,
followed by a handover of melody to the piano at bar 3,
where the piano play increasingly dissonant chords that are
eventually made up of minor seconds.\\

The piano plays this dissonant passage in triplets from bar 3 to bar 5
until it descends to the massive clusters at the beginning of bar 6.
At bar 6, the motif is accompanied by the piano playing powerful bass clusters,
which have more of an percussive effect than harmonic ones.\\

From bar 13 to 16, the piano plays chords that are made of 5 notes from
a whole tone scale. This chord raise 4 semitones from the first one
to the second one. The third chord is an octave lower than the first one,
and the forth chord is derived from raising the third chord 5 semitones.

\section{Motifs}


At bar 6, the woodwinds and strings play a modified version
of the opening motif, changed in a way that the motif
is made of clusters built in major seconds.
At bars 10 to 12, the piano plays a descending motif,
where all the chords are made of two notes that are a major seventh apart.
While the rest plays a three bar motif that is constructed by a minim,
followed by a quaver that is displaced a quavers backwards each bar.
The dynamics in these 3 bars resembles waves--whenever the piano plays forte,
the rest plays softly and vice versa.\\

In the following 4 bars, the piano support the triplet motif that the rest is
playing. That motif is made up of chord of minor seconds which oscillates up
and down for three sets of triplets and then goes downwards for the first two
bars, just descending triplets at the third bar, and descending triplets
that are grouped in twos at the forth.

\section{Structure}

\begin{tabular}{|c|c|c|}
    \hline
    Sections&Bars&Properties\\
    \hline
    A&1-2&\\
    \hline
    B&3-5&\\
    \hline
    A\(^\prime\)&6-9&\\
    \hline
    C&10-12&\\
    \hline
    D&13-16&\\
    \hline
    E&17-20&\\
    \hline
    F&21-24&\\
    \hline
    G&25-26&\\
    \hline
    G\(^\prime\)&27-31&\\
    \hline
    A\(^{\prime\prime}\)&32-36&\\
    \hline
\end{tabular}

\end{document}
