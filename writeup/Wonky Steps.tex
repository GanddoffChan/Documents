\documentclass[a4paper]{article}
\usepackage{geometry}
\title{Write-up on {\bf Wonky Steps} (draft 5)\\
(composition technique: rhythmic counterpoint)}
\author{}
\date{}

\begin{document}

\maketitle

\section{Compositional Approach}
The piece, {\bf Wonky Steps}, explores the use of rhythmic counterpoint to create a sense of instability in the music, in addition to metric displacements and polyrhythms.

\section{Structure and Motive Development}
Overall, the work is structured in 4 parts as such:

\begin{center}
	\def\arraystretch{1.5}
\begin{tabular}{|c|c|c|}
	\hline
	Section&Bars&Main Musical Features\\
	\hline
	A&1-11&falling fifth motive and cyclic bass line\\
	\hline
	B&12-16&counterpoint between bass and piano; running notes in clarinet\\
	\hline
	C&17-26&piano rhythm with increasing intensity; reprise of opening motive\\
	\hline
	A\(^\prime\)&27-35&canon in all parts; derived from the retrograded opening motive\\
	\hline
\end{tabular}
\end{center}

The piece opens with a series of falling fifths in both the piano and the double bass. After that, the double bass plays a bass line based on the melody in the clarinet part. This rhythm of the clarinet melody is constantly imitated by the piano throughout bars 5 to 10. A bar of transition is used to change the rhythmic character at bar 12, to link to a new section. The upper melody in bar 12 played by the clarinet is first played by the piano at bar 11 to foreshadow the new rhythmic motif.\\

The main theme in the second section is taken by the double bass, in a polyrhythmic fashion, against the piano part. The double bass plays the same phrase of repeating rhythm as the phrase played in unison at the first bar of the section. On the other hand, the piano is repeating another seven-beat rhythm that emphasises all the even quavers of a bar while right hand plays three quavers in unison with the double bass at the end of each bar. The three quavers at the end emphasises the interaction between the piano and double bass parts. The clarinet then plays an rising Hijaz scale from G4 all the way up to E6. It then plays a falling scale of the pattern: whole tone, whole tone, whole tone and semitone, from E6 all the way down to F\(\sharp\)3, landing on the third section.\\

The third section spans from bar 17 to bar 26. In this section, the double bass repeats yet another rhythm while outlining the chordal qualities of the bars. The main effect created in this section however comes from the piano part. The piano plays an ever increasingly intense rhythm as the music progresses. In particular, in the first bar of the section, the rhythm consists of two crochets and two dotted quavers. In the second bar, it changes to three triplets and two dotted quavers. This pattern continues until on the fourth bar, there are five quintuplets and two dotted quavers. The frequency of note heard increases, building up to the climax. As a result, a sense of increasing excitement is created as the space between each note decreases as the music progresses. In the subsequent part, the three instruments are playing in unison to an irregular rhythmic pattern to resolve to an F\(\sharp\) major chord. Every note other than the last three notes in this section are accented so that the rhythmic qualities of this motive can be projected clearly. The rhythm used here consists of 2 groups of 4 quadruplets and 2 groups of 4 regular quavers. It is also worthy to note that these groups are arranged with non-retrogradable rhythm.\\

The forth and final section starts from bar 25 until the end of the work. Canon is used in this section. It consists of a series of rising fifth motif with the rhythm derived from retrograding the opening motif. The canon resolves to a quasi-major chord.

\section{Rhythmic Technique Explored}
Rhythmic counterpoint is explored throughout the work. For example, from bar 3 to bar 11 as the double bass plays a two-bar rhythmic cycle, it creates a rhythmic counterpoint between the bass line and the upper voices. A similar approach is used from bar 13 to 20, where the bass plays repeated polyrhythmic motives while piano and clarinet playing other higher voices.\\

Other techniques are also implemented, such as a metric displacement in the piano's line from bar 5 to bar 8 where semiquavers rests are constantly being added each time as the line repeated, and a non-retrogradable rhythm from bar 21 to 22.
\end{document}
