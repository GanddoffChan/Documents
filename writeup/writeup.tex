\documentclass{article}
\usepackage[a4paper, margin=1in]{geometry}
\usepackage{graphicx}
\usepackage{pdfpages}

\begin{document}

\begin{center}
\topskip0pt
\vspace*{\fill}
\LARGE
    {\bf Composition Portfolio\\
    (Duplicate)}
\vspace*{\fill}
%
\end{center}
\begin{center}
    \large
\begin{tabular}{ll}
    Name: &Chan Lok Hin Gordon\\
    NRIC: &T0277655Z\\
    Centre/Index no.: &3042/0240\\
    School Name: &Dunman High School\\
    Subject Name: &Higher 2 Music:\\
    &Component 32: Music Writing (Minor)\\
    Subject Code: &9753/32
\end{tabular}
\end{center}

\newpage

\tableofcontents

\newpage

\addcontentsline{toc}{part}{Part 1: Composition Techniques}
\begin{center}
\topskip0pt
\vspace*{\fill}
\LARGE
    \section{Wonky Steps (rhythmic counterpoint)}
    \includegraphics[width=\textwidth]{wonk.jpg}
\vspace*{\fill}
%
\end{center}

\newpage

\subsection{Working Timeline}
\begin{center}
	\def\arraystretch{1.5}
\begin{tabular}{|c|l|l|}
	\hline
	Draft&Date&Area explored / Changes made\\
	\hline
	1&October 2020&
    descending fifth motive, metric displacement, canon\\
	\hline
	2&
    January 2021&extent of rhythmic counterpoint, sense of shifting of space and time,\\
    &&musical direction, notions, imitations between clarinet and piano\\
	\hline
	3&March 2021&harmonic progressions, sense of speeding up\\
	\hline
	Final&May 2021&sense of suspense before resolution, harmonic direction\\
    &&and harmonic support, counterpoint, sense of arrival, notations\\
	\hline
\end{tabular}
\end{center}

\subsection{Write-up on {\bf Wonky Steps}}

\subsubsection{Compositional Approach}
The piece, {\bf Wonky Steps}, explores the use of rhythmic counterpoint to
create a sense of instability in the music, in addition to metric displacements
and poly-rhythms.\\

\subsubsection{Structure and Motive Development}
Overall, the work is structured in 4 parts as such:\\

\begin{center}
	\def\arraystretch{1.5}
\begin{tabular}{|c|c|c|}
	\hline
	subsection&Bars&Main Musical Features\\
	\hline
	A&1-11&falling fifth motive and cyclic bass line\\
	\hline
	B&12-16&counterpoint between bass and piano; running notes in clarinet\\
	\hline
	C&17-26&piano rhythm with increasing intensity; reprise of opening motive\\
	\hline
	A\(^\prime\)&27-35&canon in all parts; derived from the retrograded opening motive\\
	\hline
\end{tabular}
\end{center}

The piece opens with a series of falling fifths in both the piano and the
double bass. After that, the double bass plays a bass line based on the melody
in the clarinet part. This rhythm of the clarinet melody is constantly imitated
by the piano throughout bars 5 to 10. A bar of transition is used to change the
rhythmic character at bar 12, to link to a new subsection. The upper melody in
bar 12 played by the clarinet is first played by the piano at bar 11 to
foreshadow the new rhythmic motif.\\

The main theme in the second subsection is taken by the double bass, in a
poly-rhythmic fashion, against the piano part. The double bass plays the same
phrase of repeating rhythm as the phrase played in unison at the first bar of
the subsection. On the other hand, the piano is repeating another seven-beat
rhythm that emphasises all the even quavers of a bar while right hand plays
three quavers in unison with the double bass at the end of each bar. The three
quavers at the end emphasises the interaction between the piano and double bass
parts. The clarinet then plays an rising Hijaz scale from G4 all the way up to
E6. It then plays a falling scale of the pattern: whole tone, whole tone, whole
tone and semitone, from E6 all the way down to F\(\sharp\)3, landing on the
third subsection.\\

The third subsection spans from bar 17 to bar 26. In this subsection, the
double bass repeats yet another rhythm while outlining the chordal qualities of
the bars. The main effect created in this subsection however comes from the
piano part. The piano plays an ever increasingly intense rhythm as the music
progresses. In particular, in the first bar of the subsection, the rhythm
consists of two crochets and two dotted quavers. In the second bar, it changes
to three triplets and two dotted quavers. This pattern continues until on the
fourth bar, there are five quintuplets and two dotted quavers. The frequency of
note heard increases, building up to the climax. As a result, a sense of
increasing excitement is created as the space between each note decreases as
the music progresses. In the subsequent part, the three instruments are playing
in unison to an irregular rhythmic pattern to resolve to an F\(\sharp\) major
chord. Every note other than the last three notes in this subsection are
accented so that the rhythmic qualities of this motive can be projected
clearly. The rhythm used here consists of two groups of four quadruplets and two
groups of four regular quavers. It is also worthy to note that these groups are
arranged with non-retro-gradable rhythm.\\

The forth and final subsection starts from bar 25 until the end of the work.
Canon is used in this subsection. It consists of a series of rising fifth motif
with the rhythm derived from retrograding the opening motif. The canon resolves
to a quasi-major chord.\\

\subsubsection{Rhythmic Technique Explored} Rhythmic counterpoint is explored
throughout the work. For example, from bar 3 to bar 11 as the double bass plays
a two-bar rhythmic cycle, it creates a rhythmic counterpoint between the bass
line and the upper voices. A similar approach is used from bar 13 to 20, where
the bass plays repeated poly-rhythmic motives while piano and clarinet playing
other higher voices.\\

Other techniques are also implemented, such as a metric displacement in the
piano's line from bar 5 to bar 8 where semiquavers rests are constantly being
added each time as the line repeated, and a non-retro-gradable rhythm from bar
21 to 22.\\

\newpage
\begin{center}
\topskip0pt
\vspace*{\fill}
\LARGE
\subsection{Draft 1}
\vspace*{\fill}
%
\end{center}
\newpage
\begin{center}
\topskip0pt
\vspace*{\fill}
\LARGE
\subsection{Draft 2}
\vspace*{\fill}
%
\end{center}
\newpage
\begin{center}
\topskip0pt
\vspace*{\fill}
\LARGE
\subsection{Draft 3}
\vspace*{\fill}
%
\end{center}
\newpage
\begin{center}
\topskip0pt
\vspace*{\fill}
\LARGE
\subsection{Final Version}
\vspace*{\fill}
%
\end{center}
\includepdf[scale=.9,page=-,pagecommand={\thispagestyle{plain}}]{wonky_steps.pdf}

\newpage

\begin{center}
\topskip0pt
\vspace*{\fill}
\LARGE
    \section{Infree (tone clusters)}
    \includegraphics[width=\textwidth]{infree.png}
\vspace*{\fill}
%
\end{center}

\newpage

\subsection{Working Timeline}
\begin{center}
	\def\arraystretch{1.5}
\begin{tabular}{|c|l|l|}
	\hline
	Draft&Date&Area explored / Changes made\\
	\hline
	1&October 2020
    &characteristic rhythmic motifs and ideas of progression\\
	\hline
	2&January 2021
    &rewrites of parts considering practicality, harmonic counterpoints and\\
    &&textural considerations, notations for ease for reading\\
	\hline
	3&March 2021
    &\\
	\hline
	Final&May 2021&\\
	\hline
\end{tabular}
\end{center}

\subsection{Write-up on {\bf Infree}}

\subsubsection{Introduction}

The title of \textbf{Infree} describes the state of not being able to be free.
In this case, it describes the notes and voices in the piece being somewhat
strangled to one another, failed to separate and hence not free.  It also
refers to the effect of tone clusters being rigid and not scattered. Certain
dynamic changes and rhythms are used to emphasise the quality of the
clusters.\\

\subsubsection{Harmony}

In the introductory motif, a E minor tonality is suggested by the flute and
violin.  That somewhat stable tonality is immediately destroyed at bar 2 when
the tone clusters are introduced.  The following bar are tone clusters made up
of major seconds, followed by a handover of melody to the piano at bar 3, where
the piano play increasingly dissonant chords that are eventually made up of
minor seconds.\\

The piano plays this dissonant passage in triplets from bar 3 to bar 5 until it
descends to the massive clusters at the beginning of bar 6.  At bar 6, the
motif is accompanied by the piano playing powerful bass clusters, which have
more of an percussive effect than harmonic ones.\\

From bar 13 to 16, the piano plays chords that are made of five notes from a whole
tone scale. This chord raise four semitones from the first one to the second one.
The third chord is an octave lower than the first one, and the forth chord is
derived from raising the third chord five semitones.\\

\subsubsection{Motifs}


At bar 6, the woodwinds and strings play a modified version of the opening
motif, changed in a way that the motif is made of clusters built in major
seconds.  At bars 10 to 12, the piano plays a descending motif, where all the
chords are made of two notes that are a major seventh apart.  While the rest
plays a three bar motif that is constructed by a minim, followed by a quaver
that is displaced a quavers backwards each bar.  The dynamics in these three bars
resembles waves--whenever the piano plays forte, the rest plays softly and vice
versa.\\

In the following four bars, the piano supports the triplet motif that the rest is
playing. That motif is made up of chord of minor seconds which oscillates up
and down for three sets of triplets and then goes downwards for the first two
bars, just descending triplets at the third bar, and descending triplets that
are grouped in twos at the forth.\\

\newpage
\begin{center}
\topskip0pt
\vspace*{\fill}
\LARGE
\subsection{Draft 1}
\vspace*{\fill}
%
\end{center}
\newpage
\begin{center}
\topskip0pt
\vspace*{\fill}
\LARGE
\subsection{Draft 2}
\vspace*{\fill}
%
\end{center}
\newpage
\begin{center}
\topskip0pt
\vspace*{\fill}
\LARGE
\subsection{Draft 3}
\vspace*{\fill}
%
\end{center}
\newpage
\begin{center}
\topskip0pt
\vspace*{\fill}
\LARGE
\subsection{Final Version}
\vspace*{\fill}
%
\end{center}
\includepdf[scale=0.9,page=-,pagecommand={\thispagestyle{plain}}]{infree.pdf}

\begin{center}
\topskip0pt
\vspace*{\fill}
\LARGE
    \section{Space Dust (twelve-tone serialism)}
    \includegraphics[width=\textwidth]{dust.jpg}
\vspace*{\fill}
%
\end{center}

\newpage

\subsection{Working Timeline}
\begin{center}
	\def\arraystretch{1.5}
\begin{tabular}{|c|l|l|}
	\hline
	Draft&Date&Area explored / Changes made\\
	\hline
	1&October 2020&
    came up with the twelve-tone matrix, motifs\\
	\hline
	2&January 2021&
    notations, linkage of ideas, momentum, new themes, distinctive\\
    &&musical character\\
	\hline
	3&March 2021&\\
	\hline
	Final&May 2021&\\
	\hline
\end{tabular}
\end{center}

\subsection{Write-up on {\bf Space Dust}}

\subsubsection{Compositional Approach}

The piece \textbf{Space Dust} utilises the twelve-tone technique approach. It
depicts the chaotic nature of space dust but also the birth of new stars though
the chaotic processes. This approach was revised and improved by Arnold
Schőnberg in the early twentieth century. The emphasis on all twelve tones in a
Twelve-tone Equal Temperament system being equal, with no superior pitch or a
sense of home key makesthe music very atonal.\\

\subsubsection{Harmonic Structure and Motivic Development}

The piece has two major parts, characterised by their individual tone row
matrix. The first part is in
\(\def\arraystretch{0.0}\begin{array}{c}4\\4\end{array}\), bass clarinet and
oboe play in parallel fifths, followed by a piano solo. In the second part,
the oboe and piano play dotted quaver motif, with syncopation in between them.
At the same time, the bass clarinet plays crochets on the beats, emphasising
the meter of \(\def\arraystretch{0}\begin{array}{c}6\\8\end{array}\).  The
subsection in the first four bars ends with a resolution to an A\(\flat\)
major chord. A transitional motif in bar 5 uses uniformed semiquavers. The
motive in bar 5 is constructed with lines that are three semitones apart, resulting
in twelve diminished-seven chords.  The following bars are very dissonant to emphasise
on the chaos.\\

\subsubsection{Musical Effects}

The effect of chaos is achieved through the use of poly-rhythms. One prime
example can be found from bar 21 to 23, where the oboe plays twelve evenly spaced
notes per bar, the piano four and the bass clarinet five. This creates a twelve against five
against four poly-rhythm which sounds chaotic. This highlights the chaotic
reactions of space dust, full of dis-coordinated release of heat and light.\\

The effect of consonance is achieved at the later half of the composition,
where all three instruments play the same tone row but at different amounts of
delays. Since the tone row is mostly made of fifths, the delay actually makes
many intervals of fifths between the different instruments, resulting in
consonant sounding chords. This signifies resolution and the end of chaotic
space dust reaction, meaning the birth of a new star.\\

\subsubsection{Technique Explored}

A tone row of D, G, E, A, B\(\flat\), E\(\flat\), C, F, A\(\flat\), D\(\flat\),
F\(\sharp\) and B is used. This tone row is put into a twelve-tone matrix to
generate a total of forty-eight other tone rows, which include the transposed,
retrograde, inverted and retrograde-inverted versions of the original tone row.
The tone row used is shown below:\\

\[\def\arraystretch{1.5}
\begin{array}{|c|c|c|c|c|c|c|c|c|c|c|c|c|c|}
\hline
&\mathbf{I}_0&\mathbf{I}_5&\mathbf{I}_2&\mathbf{I}_7&\mathbf{I}_8&\mathbf{I}_1&\mathbf{I}_{10}&\mathbf{I}_3&\mathbf{I}_6&\mathbf{I}_{11}&\mathbf{I}_4&\mathbf{I}_9&\\
\hline
\mathbf{P}_0&\mathrm{D}&\mathrm{G}&\mathrm{E}&\mathrm{A}&\mathrm{B}\flat&\mathrm{E}\flat&\mathrm{C}&\mathrm{F}&\mathrm{A}\flat&\mathrm{D}\flat&\mathrm{F}\sharp&\mathrm{B}&\mathbf{R}_0\\
\hline
\mathbf{P}_7&\mathrm{A}&\mathrm{D}&\mathrm{B}&\mathrm{E}&\mathrm{F}&\mathrm{B}\flat&\mathrm{G}&\mathrm{C}&\mathrm{E}\flat&\mathrm{A}\flat&\mathrm{D}\flat&\mathrm{F}\sharp&\mathbf{R}_7\\
\hline
\mathbf{P}_{10}&\mathrm{C}&\mathrm{F}&\mathrm{D}&\mathrm{G}&\mathrm{A}\flat&\mathrm{D}\flat&\mathrm{B}\flat&\mathrm{E}\flat&\mathrm{F}\sharp&\mathrm{B}&\mathrm{E}&\mathrm{A}&\mathbf{R}_{10}\\
\hline
\mathbf{P}_5&\mathrm{G}&\mathrm{C}&\mathrm{A}&\mathrm{D}&\mathrm{E}\flat&\mathrm{A}\flat&\mathrm{F}&\mathrm{B}\flat&\mathrm{D}\flat&\mathrm{F}\sharp&\mathrm{B}&\mathrm{E}&\mathbf{R}_5\\
\hline
\mathbf{P}_4&\mathrm{F}\sharp&\mathrm{B}&\mathrm{A}\flat&\mathrm{D}\flat&\mathrm{D}&\mathrm{G}&\mathrm{E}&\mathrm{A}&\mathrm{C}&\mathrm{F}&\mathrm{B}\flat&\mathrm{E}\flat&\mathbf{R}_4\\
\hline
\mathbf{P}_{11}&\mathrm{D}\flat&\mathrm{F}\sharp&\mathrm{E}\flat&\mathrm{A}\flat&\mathrm{A}&\mathrm{D}&\mathrm{B}&\mathrm{E}&\mathrm{G}&\mathrm{C}&\mathrm{F}&\mathrm{B}\flat&\mathbf{R}_{11}\\
\hline
\mathbf{P}_2&\mathrm{E}&\mathrm{A}&\mathrm{F}\sharp&\mathrm{B}&\mathrm{C}&\mathrm{F}&\mathrm{D}&\mathrm{G}&\mathrm{B}\flat&\mathrm{E}\flat&\mathrm{A}\flat&\mathrm{D}\flat&\mathbf{R}_2\\
\hline
\mathbf{P}_9&\mathrm{B}&\mathrm{E}&\mathrm{D}\flat&\mathrm{F}\sharp&\mathrm{G}&\mathrm{C}&\mathrm{A}&\mathrm{D}&\mathrm{F}&\mathrm{B}\flat&\mathrm{E}\flat&\mathrm{A}\flat&\mathbf{R}_9\\
\hline
\mathbf{P}_6&\mathrm{A}\flat&\mathrm{D}\flat&\mathrm{B}\flat&\mathrm{E}\flat&\mathrm{E}&\mathrm{A}&\mathrm{F}\sharp&\mathrm{B}&\mathrm{D}&\mathrm{G}&\mathrm{C}&\mathrm{F}&\mathbf{R}_6\\
\hline
\mathbf{P}_1&\mathrm{E}\flat&\mathrm{A}\flat&\mathrm{F}&\mathrm{B}\flat&\mathrm{B}&\mathrm{E}&\mathrm{D}\flat&\mathrm{F}\sharp&\mathrm{A}&\mathrm{D}&\mathrm{G}&\mathrm{C}&\mathbf{R}_1\\
\hline
\mathbf{P}_8&\mathrm{B}\flat&\mathrm{E}\flat&\mathrm{C}&\mathrm{F}&\mathrm{F}\sharp&\mathrm{B}&\mathrm{A}\flat&\mathrm{D}\flat&\mathrm{E}&\mathrm{A}&\mathrm{D}&\mathrm{G}&\mathbf{R}_8\\
\hline
\mathbf{P}_3&\mathrm{F}&\mathrm{B}\flat&\mathrm{G}&\mathrm{C}&\mathrm{D}\flat&\mathrm{F}\sharp&\mathrm{E}\flat&\mathrm{A}\flat&\mathrm{B}&\mathrm{E}&\mathrm{A}&\mathrm{D}&\mathbf{R}_3\\
\hline
&\mathbf{RI}_0&\mathbf{RI}_5&\mathbf{RI}_2&\mathbf{RI}_7&\mathbf{RI}_8&\mathbf{RI}_1&\mathbf{RI}_{10}&\mathbf{RI}_3&\mathbf{RI}_6&\mathbf{RI}_{11}&\mathbf{RI}_4&\mathbf{RI}_9&\\
\hline
\end{array}\]\\

The three instruments, oboe, piano and bass clarinet are involved in this
arrangement, with each of them playing different tone rows. Sometimes, the left
hand and right hand of the piano play different tone rows. The notes they play
can sometimes match and produce consonant sounding chords.\\

\newpage
\begin{center}
\topskip0pt
\vspace*{\fill}
\LARGE
\subsection{Draft 1}
\vspace*{\fill}
%
\end{center}
\newpage
\begin{center}
\topskip0pt
\vspace*{\fill}
\LARGE
\subsection{Draft 2}
\vspace*{\fill}
%
\end{center}
\newpage
\begin{center}
\topskip0pt
\vspace*{\fill}
\LARGE
\subsection{Draft 3}
\vspace*{\fill}
%
\end{center}
\newpage
\begin{center}
\topskip0pt
\vspace*{\fill}
\LARGE
\subsection{Final Version}
\vspace*{\fill}
%
\end{center}
\includepdf[scale=0.9,page=-,pagecommand={\thispagestyle{plain}}]{space_dust.pdf}

\addcontentsline{toc}{part}{Part 2: Composition}

\begin{center}
\topskip0pt
\vspace*{\fill}
\LARGE
    \section{Axiomatic Approximation (for Clarinet Quintet)}
    \includegraphics[width=\textwidth]{approximation.png}
\vspace*{\fill}
\end{center}

\newpage

\subsection{Working Timeline}
\begin{center}
	\def\arraystretch{1.5}
\begin{tabular}{|c|l|l|}
	\hline
	Draft&Date&Area explored / Changes made\\
	\hline
	1&October 2020&\\
	\hline
	2&January 2021&\\
	\hline
	3&March 2021&\\
	\hline
	Final&May 2021&\\
	\hline
\end{tabular}
\end{center}

\subsection{Write-up on {\bf Axiomatic Approximation}}

The title of \textbf{Axiomatic Approximation} comes from the fundamental axioms
in mathematics and the usual idea of approximation in science. There are three
main sections in the work. The first section is a strange waltz in
\(\def\arraystretch{0.0}\begin{array}{c}5\\8\end{array}\). The second section
    is a waltz in \(\def\arraystretch{0.0}\begin{array}{c}6\\8\end{array}\).
        The third section is a swing beat in
\(\def\arraystretch{0.0}\begin{array}{c}4\\4\end{array}\).  The orchestration
    of the work is that of a clarinet quintet, however, unlike the classical
instrumentation where there are two violins and no double bass, I have reduced
one violin in exchange of a double bass, as I feel this there should be a
higher presence of lower register sounds in my work. It is worthy to note that,
the tempos of each section are adjusted in such a way that the pulse is
constant throughout the whole piece. In particular, the pulse is maintained at
69 B.P.M..\\

The first section of the work portray a light-hearted feel. Starting with
cascading pizzicato, the five-beat feel is established at the start of the
section. The pizzicato of the strings may often also be associated with a sense
of cuteness, which sets the mood for the whole section. This section is
referred to as `strange waltz' because five beat is one beat less than the
multiple of three, six. In terms of harmony and key centres, they are rather
unstable to suit the atmosphere. For instance, the piece starts in the key of F
major. However, in bar 11, it is quickly changed to the key of F\(\sharp\)
major. The modulation here utilises the melody as a pivot, as the melody moves
down chromatically from C to B\(\flat\). After that, the same theme is played
in B\(\flat\) with clarinet rhythmic embellishments. The original theme is
played once again by the clarinet and higher strings in a tutti fashion. The
same modulation is repeated from the key of F\(\sharp\) major to the key of G
major. An interruption, however, changes in key abruptly to F major at bar 30
for two bars, and at 32 abruptly to F\(\sharp\) major. This
G\(\to\)F\(\to\)F\(\sharp\) movement is inspired by chromatic enclosure where
the target key is F\(\sharp\) but before reaching the target, the keys a
semitone around the target key is first played. At rehearsal mark C, a chord
progression of
\(\flat\)VII\(\to\)\(\flat\)iii\(\to\)VI\(\to\)ii\(\to\)V\(\to\)I then take the
key from F\(\sharp\) to G. Between the transition of section one and section
two, there is chromatic movements in the strings to modulate from F\(\sharp\)
major to D minor.\\

This second section is a waltz in compound time. It begins with a opening
melody that outlines the overall tonal centre of the whole section. This
opening portion span ten bars, establishing the key of D minor. Then, the cello
and double bass plays in parallel fifths an ostinato pattern that highlights
heavily on the compound meter feel. It also reinforce the harmony. After the
background of the theme has been well introduced into the music, the clarinet
starts play the main melody of this section, accompanied by violin and viola.
The pizzicato played by the upper strings outline the meter and suggest a
harmonic progression which agrees with the melody played by the clarinet. After
the whole melody is played once, the melody is then taken over by the upper
strings, with lower strings playing long bass notes instead of downbeat
quavers. With the melody reiterated once more, the clarinet plays flourishes on
top of the original melody in the form of virtuosic fast scales. These scales
are D Hijaz Kar ascending and D Locrian descending. An interruption soon
follows as the melody is cut off by an abrupt entry of the cello ostinato
pattern. This interruption is responded by the other instruments gradually
until rehearsal mark F where the clarinet, upper strings and lower strings are
each playing phrases that have separate lengths that converge bar 78. The
following two bars, outlining F\(\sharp\), D, B and G in the bass, act as a
cushion to get everyone ready for rehearsal mark G, which leads to the end of
the waltz.\\

The last section is a transition leading into a fast swung passage. The transition
is opened by the double bass playing the note G and the clarinet imitating the
melody from the previous section, followed by a characteristic arpeggio that land
on the note G, the same note that the double bass is playing, just four octaves
higher. After that, the violin, viola and cello play fragments of four

\newpage
\begin{center}
\topskip0pt
\vspace*{\fill}
\LARGE
\subsection{Draft 1}
\vspace*{\fill}
%
\end{center}
\newpage
\begin{center}
\topskip0pt
\vspace*{\fill}
\LARGE
\subsection{Draft 2}
\vspace*{\fill}
%
\end{center}
\newpage
\begin{center}
\topskip0pt
\vspace*{\fill}
\LARGE
\subsection{Draft 3}
\vspace*{\fill}
%
\end{center}
\newpage
\begin{center}
\topskip0pt
\vspace*{\fill}
\LARGE
\subsection{Final Version}
\vspace*{\fill}
%
\end{center}
\includepdf[scale=0.9,page=-,pagecommand={\thispagestyle{plain}}]{axiomatic_approximation.pdf}

\section{Acknowledgement}
\section{Epilogue}

\end{document}
