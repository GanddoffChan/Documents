\documentclass[a4paper]{article}
\usepackage[]{geometry}
\title{Twelve-tone Composition (draft 3): \textbf{Space Dust}}
\author{Gordon Chan}
\begin{document}
\maketitle
\section{Compositional Approach}

The piece \textbf{Space Dust} utilises the twelve-tone technique approach. It
depicts the chaotic nature of space dust but also the birth of new stars though
the chaotic processes. This approach was revised and improved by Arnold
Schongberg in the early 20th century. The emphasis on all twelve tones in a
12-TET system being equal, with no superior pitch or a sense of home key makes
the music very atonal.

\section{Harmonic Structure and Motivic Development}

The piece has to major parts, characterised by their individual tone row
matrix. The first part is in \(\begin{array}{c}4\\4\end{array}\), bass clarinet and oboe play in parallel fifths,
followed by a piano solo. In the second part, the oboe and piano play dotted
quaver motif, with syncopation in between them. At the same time, the bass
clarinet plays crochets on the beats, emphasising the meter of
\(\begin{array}{c}6\\8\end{array}\).  The section in the first four bars ends
	with a resolution to an A flat major chord. A transitional motif in bar
	5 uses uniformed semiquavers. The motive in bar 5 is constructed with
	lines that are 3 semitones apart, resulting in 12 diminished 7 chords.
	The following bars are very dissonant to emphasise on the chaos.

\section{Musical Effects}

The effect of chaos is achieved through the use of polyrhythms. One prime example can be found from bar 21 to 23, where the oboe plays 12 evenly spaced notes per bar, the piano 4 and the bass clarinet 5. This creates a 12 against 5 against 4 polyrhythm which sounds chaotic. This highlights the chaotic reactions of space dust, full of discoordinated release of heat and light.
\\
\\
The effect of consonance is achieved at the later half of the composition, where all three instruments play the same tone row but at different amounts of delays. Since the tone row is mostly made of fifths, the delay actually makes many intervals of fifths between the different instruments, resulting in consonant sounding chords. This signifies resolution and the end of chaotic space dust reaction, meaning the birth of a new star.

\section{Technique Explored}

A tone row of D, G, E, A, B\(\flat\), E\(\flat\), C, F, A\(\flat\), D\(\flat\), F\(\sharp\) and B is used. This tone row is put into a twelve-tone matrix to generate a total of 48 other tone rows, which include the transposed, retrograde, inverted and retrograde-inverted versions of the original tone row. The tone row used is shown below:
\\
\[\def\arraystretch{1.5}
\begin{array}{|c|c|c|c|c|c|c|c|c|c|c|c|c|c|}
\hline
&\mathbf{I}_0&\mathbf{I}_5&\mathbf{I}_2&\mathbf{I}_7&\mathbf{I}_8&\mathbf{I}_1&\mathbf{I}_{10}&\mathbf{I}_3&\mathbf{I}_6&\mathbf{I}_{11}&\mathbf{I}_4&\mathbf{I}_9&\\
\hline
\mathbf{P}_0&\mathrm{D}&\mathrm{G}&\mathrm{E}&\mathrm{A}&\mathrm{B}\flat&\mathrm{E}\flat&\mathrm{C}&\mathrm{F}&\mathrm{A}\flat&\mathrm{D}\flat&\mathrm{F}\sharp&\mathrm{B}&\mathbf{R}_0\\
\hline
\mathbf{P}_7&\mathrm{A}&\mathrm{D}&\mathrm{B}&\mathrm{E}&\mathrm{F}&\mathrm{B}\flat&\mathrm{G}&\mathrm{C}&\mathrm{E}\flat&\mathrm{A}\flat&\mathrm{D}\flat&\mathrm{F}\sharp&\mathbf{R}_7\\
\hline
\mathbf{P}_{10}&\mathrm{C}&\mathrm{F}&\mathrm{D}&\mathrm{G}&\mathrm{A}\flat&\mathrm{D}\flat&\mathrm{B}\flat&\mathrm{E}\flat&\mathrm{F}\sharp&\mathrm{B}&\mathrm{E}&\mathrm{A}&\mathbf{R}_{10}\\
\hline
\mathbf{P}_5&\mathrm{G}&\mathrm{C}&\mathrm{A}&\mathrm{D}&\mathrm{E}\flat&\mathrm{A}\flat&\mathrm{F}&\mathrm{B}\flat&\mathrm{D}\flat&\mathrm{F}\sharp&\mathrm{B}&\mathrm{E}&\mathbf{R}_5\\
\hline
\mathbf{P}_4&\mathrm{F}\sharp&\mathrm{B}&\mathrm{A}\flat&\mathrm{D}\flat&\mathrm{D}&\mathrm{G}&\mathrm{E}&\mathrm{A}&\mathrm{C}&\mathrm{F}&\mathrm{B}\flat&\mathrm{E}\flat&\mathbf{R}_4\\
\hline
\mathbf{P}_{11}&\mathrm{D}\flat&\mathrm{F}\sharp&\mathrm{E}\flat&\mathrm{A}\flat&\mathrm{A}&\mathrm{D}&\mathrm{B}&\mathrm{E}&\mathrm{G}&\mathrm{C}&\mathrm{F}&\mathrm{B}\flat&\mathbf{R}_{11}\\
\hline
\mathbf{P}_2&\mathrm{E}&\mathrm{A}&\mathrm{F}\sharp&\mathrm{B}&\mathrm{C}&\mathrm{F}&\mathrm{D}&\mathrm{G}&\mathrm{B}\flat&\mathrm{E}\flat&\mathrm{A}\flat&\mathrm{D}\flat&\mathbf{R}_2\\
\hline
\mathbf{P}_9&\mathrm{B}&\mathrm{E}&\mathrm{D}\flat&\mathrm{F}\sharp&\mathrm{G}&\mathrm{C}&\mathrm{A}&\mathrm{D}&\mathrm{F}&\mathrm{B}\flat&\mathrm{E}\flat&\mathrm{A}\flat&\mathbf{R}_9\\
\hline
\mathbf{P}_6&\mathrm{A}\flat&\mathrm{D}\flat&\mathrm{B}\flat&\mathrm{E}\flat&\mathrm{E}&\mathrm{A}&\mathrm{F}\sharp&\mathrm{B}&\mathrm{D}&\mathrm{G}&\mathrm{C}&\mathrm{F}&\mathbf{R}_6\\
\hline
\mathbf{P}_1&\mathrm{E}\flat&\mathrm{A}\flat&\mathrm{F}&\mathrm{B}\flat&\mathrm{B}&\mathrm{E}&\mathrm{D}\flat&\mathrm{F}\sharp&\mathrm{A}&\mathrm{D}&\mathrm{G}&\mathrm{C}&\mathbf{R}_1\\
\hline
\mathbf{P}_8&\mathrm{B}\flat&\mathrm{E}\flat&\mathrm{C}&\mathrm{F}&\mathrm{F}\sharp&\mathrm{B}&\mathrm{A}\flat&\mathrm{D}\flat&\mathrm{E}&\mathrm{A}&\mathrm{D}&\mathrm{G}&\mathbf{R}_8\\
\hline
\mathbf{P}_3&\mathrm{F}&\mathrm{B}\flat&\mathrm{G}&\mathrm{C}&\mathrm{D}\flat&\mathrm{F}\sharp&\mathrm{E}\flat&\mathrm{A}\flat&\mathrm{B}&\mathrm{E}&\mathrm{A}&\mathrm{D}&\mathbf{R}_3\\
\hline
&\mathbf{RI}_0&\mathbf{RI}_5&\mathbf{RI}_2&\mathbf{RI}_7&\mathbf{RI}_8&\mathbf{RI}_1&\mathbf{RI}_{10}&\mathbf{RI}_3&\mathbf{RI}_6&\mathbf{RI}_{11}&\mathbf{RI}_4&\mathbf{RI}_9&\\
\hline
\end{array}\]
\\
The three instruments, oboe, piano and bass clarinet are involved in this arrangement, with each of them playing different tone rows. Sometimes, the left hand and right hand of the piano play different tone rows. The notes they play can sometimes match and produce consonant sounding chords.
\end{document}
