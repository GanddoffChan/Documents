\documentclass[a4paper,12pt]{article}
\usepackage{setspace}
\usepackage{hyperref}
\doublespacing
\date{}
\title{\bf Higher 1 Project Work (8808)\\
Insights and Reflections}
\begin{document}
\maketitle
\begin{center}
\begin{itemize}
    \item[] School: {\bf Dunman High School}
    \item[] Group Index Number: {\bf DH054}
    \item[] Group Task: {\bf 1}
    \item[] Title of Project: {\bf Technology? So EZ!}
    \item[] Word Count: {\bf 470}
    \item[] \bf CANDIDATE NAME: CHAN LOK HIN GORDON
    \item[] CANDIDATE CENTRE / INDEX NUMBER: 3041 / 1247
\end{itemize}
\end{center}
\newpage
Our Proposal aims to impart knowledge tailored to the elderly's preferences via reality show. One strength of the proposal is its design, which centred on our target audience, attracting their participation. A huge portion of the viewership of television programmes is the elderly, possibly due to the excessive spare time they have in their daily life. Reality shows such as The Joy Truck, which had been airing for 3 seasons, from 2013 to 2015 (MediaCorp, 2020)\footnote{\url{https://contentdistribution.mediacorp.sg/collections/reality/products/the-joy-truck}}, prove the appeal of the reality format, with a stable viewer base. The choice of elderly participants in the show makes it more relatable as elderly audiences can identify themselves in the show. This creates the impression of the subject matter, technology, being daunting less significant, letting the audience to be less fearful of it.\par
One challenge our proposal faces is that, our target agent, MediaCorp, although being capable of carrying out our proposed intervention of shooting and airing a reality television show, they might not be willing to do so. This is because there are a few incentives given to it that are compelling. The main concern the agent may have is the lack of profit brought by the show, making it irrational for it to adopt the proposed idea. This challenge can be overcome by finding sponsers such as SkillsFuture to fund the show. Futhermore, since our show promotes the impartation of technological knowledge to the elderly, which is first-to-none, it has great novelty. This property of our show will convince MediaCorp that the ratings of the show should not be a concern because people like to see refreshing and out-of-the-ordinary content, especially when television shows are often seen as repetitive and boring nowadays (The Atlantic, 2010)\footnote{\url{https://www.theatlantic.com/entertainment/archive/2010/09/why-every-reality-tv-show-has-almost-the-same-plot/63103/}}.\par
Our proposal can be possibly adapted to teach senior workers aged between 45 and 65 who are having difficulties in keeping up with new technologies. This group is similar to our original target audience because it also faces the threat of being eliminated by the frenetically-paced technological advancements and suffers from cognitive decline. The main difference between these two groups of people is that the senior workers do not just learn technology for their leisure but also for work and job opportunities. Therefore, the content of the show must be adapted to suit a more work-centric goal of watching the show. Moreover, the show should be uploaded online as most senior workers may not have the time to watch television and perfer watching online contents at their own convenience (Wired, 2008)\footnote{\url{https://www.wired.com/2008/09/online-tv-viewe/}}. Also, since some workers would probably want to gain knowledge and be certified, we can collaborate with SkillsFuture to provide bitesized online courses. The bite-size design ensures that the knowledge is consumed at a comprehensible pace, while online format ensures that it is both convenient and accessible, thereby enhancing the effectiveness of the courses.
\end{document}
