\documentclass[a4paper]{article}
\usepackage{amsmath,amsfonts,amsthm,amssymb}
\newcommand{\vv}{\mathbf}
\newcommand{\rref}{\mathrm{rref}}
\newcommand{\nul}{\mathrm{nul}}
\newcommand{\range}{\mathrm{range}}
\newcommand{\nullity}{\mathrm{nullity}}
\newcommand{\rank}{\mathrm{rank}}

\begin{document}
\section*1
\subsection*{(i)}

\[\begin{aligned}
	\vv A&=\begin{bmatrix}1&3&-2&a\\2&-1&3&-5\\-3&-3&0&3\end{bmatrix}\\
	     &\to\begin{bmatrix}6&18&-12&6a\\6&-3&9&-15\\-6&-6&0&6\end{bmatrix}\\
	     &\to\begin{bmatrix}1&1&0&-1\\0&-9&9&-9\\0&12&-12&6a+6\end{bmatrix}\\
	     &\to\begin{bmatrix}1&1&0&-1\\0&-2&2&-2\\0&2&-2&a+1\end{bmatrix}\\
	     &\to\begin{bmatrix}1&1&0&-1\\0&1&-1&1\\0&0&0&a-1\end{bmatrix}=\rref(\vv A)
\end{aligned}\]

Given that the dimension of the null space of \(T\) is 2, the matrix \(\vv a\) must have 2 non-pivot columns.

\[a-1=0\iff a=\boxed1\]

	\[\nul(T)=\nul(\vv A)=\nul(\rref(\vv A))=\nul\left(\begin{bmatrix}1&1&0&-1\\0&1&-1&1\\0&0&0&0\end{bmatrix}\right)\]
	\[x_4=\mu,\,x_3=\lambda,\,x_2=\lambda-\mu,\,x_1=-\lambda+\mu\]
	\[\vv x=\begin{bmatrix}-\lambda+\mu\\\lambda-\mu\\\lambda\\\mu\end{bmatrix}
	=\lambda\begin{bmatrix}-1\\1\\1\\0\end{bmatrix}
	+\mu\begin{bmatrix}1\\-1\\0\\1\end{bmatrix}\]
	Thus, a basis for the null space of \(T\) is:
	\[\boxed{\left\{\begin{bmatrix}-1\\1\\1\\0\end{bmatrix},\begin{bmatrix}1\\-1\\0\\1\end{bmatrix}\right\}}\]

\subsection*{(ii)}
\[\dim(\range(T))=\rank(\vv A)=4-\nullity(\vv A)=4-2=2\]
Thus, the range space of \(T\) is a plane.\qed
\[R:\vv r=\lambda\begin{bmatrix}1\\2\\-3\end{bmatrix}
+\mu\begin{bmatrix}3\\-1\\-3\end{bmatrix},\,\lambda,\mu\in\mathbb R\]
\[\vv n=\begin{bmatrix}3\\-1\\-3\end{bmatrix}\times\begin{bmatrix}1\\2\\-3\end{bmatrix}=\begin{bmatrix}9\\6\\7\end{bmatrix}\]
\[\vv r\cdot\vv n=0\iff\vv r\cdot\begin{bmatrix}9\\6\\7\end{bmatrix}=0\]
\[R:\boxed{9x+6y+7z=0}\]

\subsection*{(iii)}
\(R\cup V\) is a vector space implies that \(V\subset R\).
\[\vv v\cdot\vv n=0\iff\begin{bmatrix}0\\b\\c\end{bmatrix}\cdot\begin{bmatrix}9\\6\\7\end{bmatrix}=0\]
\[\boxed{6b+7c=0}\]

\section*2
If \(\vv x_1\), \(\vv x_2\) and \(\vv x_3\) are linearly dependent, then:


\[\begin{aligned}
	a\vv x_1+b\vv x_2+c\vv x_3&=\vv 0,\,a,b,c\in\mathbb R\backslash\{0\}\\
	\vv M(a\vv x_1+b\vv x_2+c\vv x_3)&=\vv M\vv 0\\
	\vv Ma\vv x_1+\vv Mb\vv x_2+\vv Mc\vv x_3&=\vv 0\\
	a\vv M\vv x_1+b\vv M\vv x_2+c\vv M\vv x_3&=\vv 0,\,a,b,c\in\mathbb R\backslash\{0\}\\
\end{aligned}\]
Therefore, \(\vv M\vv x_1\), \(\vv M\vv x_2\) and \(\vv M\vv x_3\) are linearly dependent.

\subsection*{(i)}
\[\det\left(\begin{bmatrix}\vv y_1&\vv y_2&\vv y_3\end{bmatrix}\right)=0\]
Since the matrix \(\begin{bmatrix}\vv y_1&\vv y_2&\vv y_3\end{bmatrix}\) is not full rank, the dimension of its column space is less than 3. This implies that the span of the vectors \(\vv y_1\), \(\vv y_2\) and \(\vv y_3\) has a dimension that is less than 3, which is the number of vectors. Therefore, \(\vv y_1\), \(\vv y_2\) and \(\vv y_3\) are linearly dependent.

\subsection*{(ii)}
\[\vv A\begin{bmatrix}\vv y_1&\vv y_2&\vv y_3\end{bmatrix}
=\begin{bmatrix}2&6&4\\20&10&190\\34&2&368\end{bmatrix}
\to\begin{bmatrix}1&0&11\\0&1&-3\\0&0&0\end{bmatrix}\]
A basis for the linear space spanned by the vectors \(\vv y_1\), \(\vv y_2\) and \(\vv y_3\) is:
\[\boxed{\left\{\begin{bmatrix}1\\10\\17\end{bmatrix},\begin{bmatrix}3\\5\\1\end{bmatrix}\right\}}\]

\subsection*{(iii)}
\[\nul(T)=\nul(\vv A)=\nul\left(\begin{bmatrix}-1&1&0\\0&4&2\\3&5&4\end{bmatrix}\right)=\nul\left(\begin{bmatrix}1&0&1/2\\0&1&1/2\\0&0&0\end{bmatrix}\right)\]
\[x_3=\lambda,\,x_2=-\frac12\lambda,\,x_1=-\frac12\lambda\]
\[\vv x=\begin{bmatrix}-\lambda/2\\-\lambda/2\\\lambda\end{bmatrix}
=-\frac\lambda2\begin{bmatrix}1\\1\\-2\end{bmatrix}\]
A basis for the null space of \(T\) is:
\[\boxed{\left\{\begin{bmatrix}1\\1\\-2\end{bmatrix}\right\}}\]

\subsection*{(iv)}
\[T(\vv0)=\vv0\ne\begin{bmatrix}1\\6\\9\end{bmatrix}\]
Since the zero vector is not contained by the set of solutions of the equation, it is not a vector space.

It is obvious that:
\[T\left(\begin{bmatrix}0\\1\\1\end{bmatrix}\right)=\begin{bmatrix}1\\6\\9\end{bmatrix}\]
Thus,
\[\vv x=\vv x_0+\vv x_p=k\begin{bmatrix}1\\1\\-2\end{bmatrix}+\begin{bmatrix}0\\1\\1\end{bmatrix}=\begin{bmatrix}k\\1+k\\1-2k\end{bmatrix},\,k\in\mathbb R\]
\[\begin{aligned}
	|\vv x|&=\left|\begin{bmatrix}k\\1+k\\1-2k\end{bmatrix}\right|\\
	       &=\sqrt{k^2+(1+k)^2+(1-2k)^2}\\
	       &=\sqrt{k^2+1+2k+k^2+1-4k+4k^2}\\
	       &=\sqrt{2-2k+6k^2}\\
	       &=\sqrt{2+6\left(-\frac k3+k^2\right)}\\
	       &=\sqrt{2+6\left(-\frac1{36}+\frac1{36}-\frac k3+k^2\right)}\\
	       &=\sqrt{2-\frac16+6\left(\frac16-k\right)^2}\\
	       &=\sqrt{\frac{11}6+\frac16\left(1-6k\right)^2}\\
	       &=\frac{\sqrt{6}}{6}\sqrt{11+\left(1-6k\right)^2}
\end{aligned}\]
\[\min(|\vv x|)=\frac{\sqrt6}6\sqrt{11}=\boxed{\frac{\sqrt{66}}6}\]
\end{document}
