\documentclass[]{article}
\usepackage{amsmath}
\usepackage{amsthm}
\usepackage{amssymb}

\begin{document}
\[\begin{aligned}
	\theta+40^\circ&=180^\circ\\
	\theta&=140^\circ\\
\end{aligned}\]

\[S=180^\circ(n-2)\]
\[S=n\theta\]
\\
\[\begin{aligned}
	S&=S\\
	180^\circ(n-2)&=n\theta\\
	180^\circ n-360^\circ&=n\theta\\
	180^\circ n-n\theta&=360^\circ\\
	n\left(180^\circ-\theta\right)&=360^\circ\\
	n&=\frac{360^\circ}{180^\circ-\theta}\\
\end{aligned}\]

Let the smallest angle be \(2k\). This means that the other angles will be \(3k\), \(4k\), \(4k\) and \(5k\) respectively.

Let \(S\) be the sum of the interior angles in the pentagon.
\[\begin{aligned}
	S&=S\\
	2k+3k+4k+4k+5k&=180^\circ(5-2)\\
	18k&=540^\circ\\
	k&=30^\circ
\end{aligned}\]

Therefore, since the largest angle is \(5k\), it is \(5(30^\circ)=150^\circ\).
\\
\newpage
\[\begin{aligned}
	\angle XZB+\angle ABZ&=180^\circ\,\text{(int. }\angle\text{s, AB//XZ)}\\
	59^\circ+\angle YBZ+46^\circ&=180^\circ\\
	\angle YBZ&=\boxed{75^\circ}\\
\end{aligned}\]

\[\begin{aligned}
	a+134^\circ&=180^\circ\,\text{(int. }\angle\text{s},\,AB//\ell)\\
	a&=46^\circ\,\text{(int. }\angle\text{s},\,AB//\ell)\\
\end{aligned}\]
\[\begin{aligned}
	b&=59^\circ\,\text{(alt. }\angle\text{s},\,XZ//\ell)\\
\end{aligned}\]
\[\text{obtuse }\angle CWZ=a+b=46^\circ+59^\circ=\boxed{105^\circ}\]

\[\begin{aligned}
	30^\circ+30^\circ+30^\circ+(360^\circ-x^\circ)&=360^\circ\\
	x&=\boxed{270^\circ}
\end{aligned}\]
\\
\[\begin{aligned}
    &\frac13ab+\left(\frac14ab^2-\frac15ba\right)+\left(\frac15ab^2+ab\right)\\
    &=\frac13ab+\frac14ab^2-\frac15ba+\frac15ab^2+ab\\
    &=\frac13ab-\frac15ab+ab+\frac14ab^2+\frac15ab^2\\
    &=ab\left(\frac13-\frac15+1\right)+ab^2\left(\frac14+\frac15\right)\\
    &=\frac{17}{15}ab+\frac9{20}ab^2\\
    &=ab\left(\frac{17}{15}+\frac9{20}b\right)\\
\end{aligned}\]

Commutative: \(A+B=B+A\), \(AB=BA\)
\[a-b\ne b-a,\,a-b=a+(-b)=(-b)+a\]
\[\frac ab\ne \frac ba,\,a\cdot\frac1b=\frac1b\cdot a\]

\newpage

\[\begin{aligned}
    &\left(3a+\frac14b\right)+\left(-b+\frac14c\right)+\left(-c+\frac14a\right)\\
    &=3a+\frac14b-b+\frac14c-c+\frac14a\\
    &=3a+\frac14a+\frac14b-b+\frac14c-c\\
    &=\frac{13}4a-\frac34b-\frac34c
\end{aligned}\]
\\
\[\begin{aligned}
    &\left(-\frac13a+\frac34b-\frac45\right)-\left(\frac43a-\frac85b+\frac25\right)\\
    &=-\frac13a+\frac34b-\frac45-\frac43a+\frac85b-\frac25\\
    &=-\frac13a-\frac43a+\frac34b+\frac85b-\frac45-\frac25\\
    &=-\frac53a+\frac{47}{20}b-\frac65\\
\end{aligned}\]
\\
\[-a-b=-(a+b)\]

\[\begin{aligned}
    &\frac{8(3x-4y)}5-\frac{4x-y}{10}+\frac{3(x-3)}5\\
    &=\frac{8(3x-4y)(2)}{10}-\frac{4x-y}{10}+\frac{3(x-3)(2)}{10}\\
    &=\frac{48x-64y-4x+y+6x-18}{10}\\
    &=\frac{48x-4x+6x-64y+y-18}{10}\\
    &=\frac{50x-63y-18}{10}
\end{aligned}\]

\newpage
\[x+y+2z\]
\[(60x+80y+450z)\,\mathrm{g}\]
\[36-x-y-2z\]
\\
\[\begin{aligned}
    \text{perimeter of }\triangle&=\text{perimeter of rectangle}\\
    3(x+1)&=2(x-1+y)\\
    3x+3&=2x-2+2y\\
    -2x+2&=-2x+2\\
    \hline
    x+5&=2y\\
    y&=\boxed{\frac{x+5}2}
\end{aligned}\]

\[\begin{aligned}
    5p-7&=3p+1\\
    2p&=8\\
    p&=4
\end{aligned}\]

\[DC=5p-7=5(4)-7=20-7=13\]

\[\begin{aligned}
    S&=S\\
    156^\circ n&=180^\circ(n-2)\\
    156n&=180n-360\\
    24n&=360\\
    n&=15
\end{aligned}\]

\newpage

\[\begin{aligned}
    -15+2(5-2x)&=4(2-3x)+19\\
    -15+10-4x&=8-12x+19\\
    -5-4x&=27-12x\\
    8x&=32\\
    x&=\boxed4\\
\end{aligned}\]

\[\begin{aligned}
    0.99x+1.4(22-x)&=27.52\\
    0.99x+30.8-1.4x&=27.52\\
    -0.41x+30.8&=27.52+0x\\
    30.8-27.52&=0.41x\\
    0.41x&=3.28\\
    x&=\boxed8\\
\end{aligned}\]

\[\begin{aligned}
    3&=28-\frac5{12}x\\
    36&=336-5x\\
    5x&=300\\
    x&=\boxed{60}
\end{aligned}\]

\[\begin{aligned}
    2\frac58x-28\frac38&=34\frac58\\
    \frac{21}8x-\frac{227}8&=\frac{277}8\\
    21x-227&=277\\
    21x&=504\\
    x&=\boxed{24}
\end{aligned}\]

\[\begin{aligned}
    \frac{3(x-4)}4-\frac{x+3}5&=-\frac3{10}\\
    60x-80-4x-12&=-6\\
    56x-92&=-6\\
    56x&=84\\
    x&=\boxed{\frac32}\\
\end{aligned}\]

\newpage
\[\begin{aligned}
    \frac2{9(7x-3)}&=\frac3{81x-45}\\
    \frac2{9(7x-3)}&=\frac3{9(9x-5)}\\
    1\div\frac2{7x-3}&=1\div\frac3{9x-5}\\
    \frac{7x-3}2&=\frac{9x-5}3\\
    21x-9&=18x-10\\
    3x&=-1\\
    x&=\boxed{-\frac13}\\
\end{aligned}\]

\[\begin{aligned}
    \frac2{9(7x-3)}&=\frac3{81x-45}\\
    \frac2{9(7x-3)}&=\frac3{9(9x-5)}\\
    18x-10&=21x-9\\
    -1&=3x\\
    x&=\boxed{-\frac13}\\
\end{aligned}\]

\[\begin{aligned}
    \sqrt{2x}&=\sqrt{x-1}
\end{aligned}\]
\newpage
\[p=6+4(n-1)=2+4n\]
\[q=6+2(n-1)=2(3+n-1)=2(n+2)\]

\[\begin{aligned}
    \frac5x&=10\\
    5&=10x\\
    x&=\frac12
\end{aligned}\]
\[\begin{aligned}
    \frac{y+2}2&=5\\
    y+2&=10\\
    y&=8
\end{aligned}\]
\\

Let the radius of the cylinder be \(r\).

\[\begin{aligned}
    V_{\text{cylinder}}&=24.75\pi\\
    \pi r^2(11)&=24.75\pi\\
    r^2(11)&=24.75\\
    r^2&=2.25\\
    r&=1.5\,\text{cm}\qed
\end{aligned}\]

\[\begin{aligned}
    V&=V_\text{prism}-V_\text{cylinder}\\
    &=\frac12(10)(12)(11)-24.75\pi\\
    &\approx582\,\text{cm}^3
\end{aligned}\]
\[2\times292\times110\%\times107\%\approx\$687.37\]

Option 1:
\[687.37+2\times2\times29.90\times110\%\times107\%\approx\$828.14\]
Option 2:
\[2\times342\times110\%\times107\%\times90\%\approx\$724.56\]

\[\frac{10051}{16.25}=\frac{20100}{333}\,\text{km}\,\text{h}^{-1}\]
\[\frac{1431+1431+10051}{16.25}=\frac{12913}{\frac{65}{4}}=\frac{51612}{65}\,\text{km}\,\text{h}^{-1}\]
\[\frac{20100}{333}\,\text{km}\,\text{h}^{-1}\leqslant\text{average speed}\leqslant\frac{51612}{65}\,\text{km}\,\text{h}^{-1}\]
\newpage

\[\begin{array}{|c|c|}
    \hline
    \text{grades}&\text{frequency}\\
    \hline
    A&5\\
    B&8\\
    C&10\\
    D&7\\
    E&6\\
    \hline
    \text{total}&36\\
    \hline
\end{array}\]
\[\begin{array}{|c|c|}
    \hline
    \text{rating}&\text{frequency}\\
    \hline
    2&61\\
    1&14\\
    0&24\\
    -1&57\\
    -2&45\\
    \hline
\end{array}\]
\[\frac{2(61)+1(14)+0(24)+(-1)(57)+(-2)(45)}{61+14+24+57+45}=\frac{-11}{201}\approx-0.0547\]
\end{document}
