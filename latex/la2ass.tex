\documentclass[a4paper]{article}
\usepackage{amsmath,amsfonts,amsthm,amssymb}
\usepackage{graphicx}
\usepackage{concmath}
\usepackage[T1]{fontenc}
\newcommand{\vv}{\mathbf}
\newcommand{\rank}{\operatorname{rank}}
\newcommand{\rref}{\operatorname{rref}}
\newcommand{\nullity}{\operatorname{nullity}}

\begin{document}
1.

Let \(\vv M=\begin{bmatrix}
                1&b&a\\
                a&1&b\\
                b&a&1
        \end{bmatrix}\).

\[\begin{aligned}
        \det(\vv M)&=1(1-ab)+b\left(b^2-a\right)+a\left(a^2-b\right)\\
                       &=1-ab+b^3-ab+a^3-ab\\
                       &=a^3-3ab+b^3+1
\end{aligned}\]

Since \(\left\{
        \begin{bmatrix}
        1\\a\\b
        \end{bmatrix},
        \begin{bmatrix}
        a\\1\\b
        \end{bmatrix},
        \begin{bmatrix}
        b\\a\\1
        \end{bmatrix}\right\}\)
        is not a basis for \(\mathbb R^3\), \(\operatorname{rank}(\vv M)<3\).

        Thus, \(\det(\vv M)=0\).
        \[a^3-3ab+b^3+1=0\qed\]


2. (i)

\[\vv A\vv0=\vv0\implies\vv0\in W\]
Let \(\vv u,\vv v\in W\).
(\(\implies\vv A\vv u=\vv u\land\vv A\vv v=\vv v\))
\[\vv A(a\vv u+b\vv v)
        =\vv A(a\vv u)+\vv A(b\vv v)
        =a\vv A\vv u+b\vv A\vv v
        =a\vv u+b\vv v
\implies a\vv u+b\vv v\in W\]
Since \(W\subset\mathbb R^n\), \(\vv0\in W\)
and \(\vv u,\vv v\in W\implies a\vv u+b\vv v\in W\),
\(W\) is a subspace of \(\mathbb R^n\).\qed


2. (ii)

\[\vv A\vv u=\vv u\iff\vv A\vv u-\vv I\vv u=\vv0\iff(\vv A-\vv I)\vv u=\vv0\]
\[\begin{bmatrix}0&0&-1\\0&0&0\\0&0&-2\end{bmatrix}
\begin{bmatrix}u_1\\u_2\\u_3\end{bmatrix}=0\implies u_3=0\]
Let \(u_1=\lambda\) and \(u_2=\mu\).
\[\vv u=\begin{bmatrix}\lambda\\\mu\\0\end{bmatrix}
=\lambda\begin{bmatrix}1\\0\\0\end{bmatrix}+\mu\begin{bmatrix}0\\1\\0\end{bmatrix}\]
A basis of \(W\), is
\[\boxed{\left\{\begin{bmatrix}1\\0\\0\end{bmatrix},
\begin{bmatrix}0\\1\\0\end{bmatrix}\right\}}\]


3. (i)

Let \(\vv X=\begin{bmatrix}\vv x_1&\vv x_2&\vv x_3&\vv x_4\end{bmatrix}\).
\[\det(\vv X)=1(1(1(1)))=1\ne0\implies \rank(\vv X)=4\]
Since \(\vv X\) is full rank, the columns of \(\vv X\), \(\vv x_1\), \(\vv x_2\), \(\vv x_3\) and \(\vv x_4\) form a basis of \(\mathbb R^4\).\qed


3. (ii)

\[\rref(\vv A)=\begin{bmatrix}
        1&0&-1&2\\
        0&1&2&-1\\
        0&0&0&0\\
        0&0&0&0
\end{bmatrix}
\implies\rank(A)=\boxed2\]
\[\nullity(\vv A)=4-2=\boxed2\]


3. (iii)

\[\vv A\vv x_1=\begin{bmatrix}1\\-3\\4\\6\end{bmatrix}\]
\[\vv A\vv x_2=\begin{bmatrix}2\\1\\9\\4\end{bmatrix}\]

Since \(\rank(\vv A)=2\), all \(\vv A\vv x_i\) are coplanar. Therefore, all linear combinations of these vector, ie \(\sum\lambda\vv A\vv x\), exist on the same two dimensional space. Moreover, since \(\vv A\vv x_1\) cannot be expressed as a scalar multiple of \(\vv A\vv x_2\), the dimension of \(V\) cannot be 0 or 1. Thus, \(\dim(V)=2\).\qed

A basis of \(V\), is
\[\boxed{\left\{\begin{bmatrix}1\\-3\\4\\6\end{bmatrix},
\begin{bmatrix}2\\1\\9\\4\end{bmatrix}\right\}}\]


3. (iv)

\[\begin{bmatrix}p\\q\\23\\6\end{bmatrix}=\lambda\begin{bmatrix}1\\-3\\4\\6\end{bmatrix}+\mu\begin{bmatrix}2\\1\\9\\4\end{bmatrix}\]

\[\begin{cases}
        4\lambda+9\mu=23\\
        6\lambda+4\mu=6
\end{cases}
\iff
\begin{cases}
        12\lambda+27\mu=69\\
        12\lambda+8\mu=12
\end{cases}
\implies
19\mu=57\iff \mu=3\]

\[6\lambda+4(3)=6\iff6\lambda=-6\iff\lambda=-1\]

\[p=(-1)1+(3)2=\boxed5\]
\[q=(-1)(-3)+(3)1=\boxed6\]
\end{document}
