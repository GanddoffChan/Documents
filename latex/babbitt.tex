\documentclass[a4paper]{article}
\usepackage{amsmath,amsfonts,amsthm,amssymb}
\usepackage{graphicx}
\title{Babbitt's Semi-Simple Variations}
\author{Gordon Chan}

\begin{document}
\maketitle

Like most people, I am exposed to mostly tonal music since young.
Thus, I have a great intuition for predicting and appreciating tonal music.
Semi-Simple Variations, however, is no tonal music. 
It is composed not using conventional harmony, rhythm, etc.\
but twelve-tone technique. When I first heard it in class, 
I felt lost. I did not understand what is going on. 
The musical direction was unclear to and I could not predict where the music is going.
After a quick introduction of the tools that Babbitt used to compose this piece,
I slowly understand what is really going on.
I know that the notion of variation in this case is very unique.
Instead of the usual approach of rhythmic alteration or modal interchange,
the variations in Semi-Simple Variations is generated by twelve-tone methods
such as inverse, retrograde and pitch shifting.
Knowing this allows me to listen to the music and follow the structure of it.
This finally results in me being able to appreciate it more.
Although I am unable to appreciate it intuitively, I can still attempt to 
appreciate it in its methodological and mathematical beauty.\

This is definitely still music. Anything that has a audible quality can be music if we are brave enough.
Ultimately, music is an illusion created by the brain that has no bearing
on the physical existence of sound. Thus, the notion of a climax is only
a peak of our own intellectual activity, which means that although the
form of climax and logic in Semi-Simple Variations is not obvious for most of us,
if some of us are able to understand the piece intuitively, they will be able to 
regard the piece as music easily.

\end{document}
