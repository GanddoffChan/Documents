\documentclass[a4paper]{article}
\usepackage{amsmath,amsfonts,amsthm,amssymb}
\usepackage{geometry}
\newcommand{\vv}{\mathbf}
\newcommand{\dd}{\mathrm d}
\newcommand{\spn}{\mathrm{span}}
\newcommand{\col}{\mathrm{col}}
\newcommand{\range}{\mathrm{range}}

\begin{document}

\section*{1}
\subsection*{(i)}

Let the characteristic polynomial of \(\vv A\) be \(p(\lambda)\).

\[\begin{aligned}
	p(\lambda)&=\det(\lambda\vv I-\vv A)\\
	&=\det\left(\begin{bmatrix}
			\lambda-1&-c&-3\\
			-4&\lambda-1&0\\
			-3&0&\lambda-1
	\end{bmatrix}\right)\\
	&=(\lambda-1)^2(\lambda-1)-4c(\lambda-1)-9(\lambda-1)\\
	&=(\lambda-1)\left((\lambda-1)^2-4c-9\right)\\
	&=(\lambda-1)\left(\lambda^2-2\lambda-4c-8\right)\\
	&=(\lambda-1)((\lambda-6)(\lambda+4)+16-4c)\\
	&=(\lambda+4)(\lambda-1)(\lambda-6)+(16-4c)(\lambda-1)\\
\end{aligned}\]
Given that \(\vv A\) has an eigenvalue of 6, \(p(\lambda)\) must be in the form of:
\[(\lambda-\lambda_1)(\lambda-\lambda_2)(\lambda-6)\]
Thus, \(16-4c=0\iff c=\boxed4\) and the remaining eigenvalues are
\(\boxed{-4}\) and \(\boxed1\).

\subsection*{(ii)}
\[\ker(\vv A+4\vv I)=\ker\left(\begin{bmatrix}
		5&4&3\\
		4&5&0\\
		3&0&5
		\end{bmatrix}\right)=\ker\left(\begin{bmatrix}
		1&0&5/3\\
		0&1&-4/3\\
		0&0&0
	\end{bmatrix}\right)=\left\{\mu_1\begin{bmatrix}-5\\4\\3\end{bmatrix}:\mu_1\in\mathbb R\right\}\]

\[\ker(\vv A-\vv I)=\ker\left(\begin{bmatrix}
		0&4&3\\
		4&0&0\\
		3&0&0
		\end{bmatrix}\right)=\ker\left(\begin{bmatrix}
		1&0&0\\
		0&1&3/4\\
		0&0&0
	\end{bmatrix}\right)=\left\{\mu_2\begin{bmatrix}0\\-3\\4\end{bmatrix}:\mu_2\in\mathbb R\right\}\]

\[\ker(\vv A-6\vv I)=\ker\left(\begin{bmatrix}
		-5&4&3\\
		4&-5&0\\
		3&0&-5
		\end{bmatrix}\right)=\ker\left(\begin{bmatrix}
		1&0&-5/3\\
		0&1&-4/3\\
		0&0&0
	\end{bmatrix}\right)=\left\{\mu_3\begin{bmatrix}5\\4\\3\end{bmatrix}:\mu_3\in\mathbb R\right\}\]

\[\vv P=\boxed{\begin{bmatrix}-5&0&5\\4&-3&4\\3&4&3\end{bmatrix}}\]
\[\vv D=\boxed{\begin{bmatrix}-4&0&0\\0&1&0\\0&0&6\end{bmatrix}}\]

\subsection*{(iii)}
\[\begin{aligned}
	\vv Y^\prime&=\vv A\vv Y\\
	\vv Y^\prime&=\vv P\vv D\vv P^{-1}\vv Y\\
	\vv P^{-1}\vv Y^\prime&=\vv D\vv P^{-1}\vv Y\\
	\vv U^\prime&=\vv D\vv U\qed
\end{aligned}\]

\[\begin{aligned}
	\vv U^\prime&=\vv D\vv U\\
	\begin{bmatrix}
		\dd u_1/\dd x\\
		\dd u_2/\dd x\\
		\dd u_3/\dd x
	\end{bmatrix}
		    &=
		    \begin{bmatrix}
			    -4&0&0\\
			    0&1&0\\
			    0&0&6
		    \end{bmatrix}
		    \begin{bmatrix}
			    u_1\\
			    u_2\\
			    u_3
		    \end{bmatrix}\\
	\begin{bmatrix}
		\dd u_1/\dd x\\
		\dd u_2/\dd x\\
		\dd u_3/\dd x
	\end{bmatrix}
		    &=
		    \begin{bmatrix}
			    -4u_1\\
			    u_2\\
			    6u_3
		    \end{bmatrix}\\
	\begin{bmatrix}
		\dd u_1/u_1\\
		\dd u_2/u_2\\
		\dd u_3/u_3
	\end{bmatrix}
		    &=
		    \begin{bmatrix}
			    -4\,\dd x\\
			    \,\dd x\\
			    6\,\dd x
		    \end{bmatrix}\\
	\begin{bmatrix}
		\ln|u_1|\\
		\ln|u_2|\\
		\ln|u_3|
	\end{bmatrix}
		    &=
		    \begin{bmatrix}
			    -4x+c_1\\
			    x+c_2\\
			    6x+c_3
		    \end{bmatrix}\\
	\begin{bmatrix}
		u_1\\
		u_2\\
		u_3
	\end{bmatrix}
		    &=
		    \begin{bmatrix}
			    \exp(-4x+c_1)\\
			    \exp(x+c_2)\\
			    \exp(6x+c_3)
		    \end{bmatrix}\\
		    \vv U
		    &=
		    \begin{bmatrix}
			    A_1\exp(-x)\\
			    A_2\exp(x)\\
			    A_3\exp(6x)
		    \end{bmatrix}\\
\end{aligned}\]

\[\begin{aligned}
	\vv U&=\vv P^{-1}\vv Y\\
	\vv Y&=\vv P\vv U\\
	\begin{bmatrix}
		y_1\\
		y_2\\
		y_3
	\end{bmatrix}&=
	\begin{bmatrix}
		-5&0&5\\
		4&-3&4\\
		3&4&3
	\end{bmatrix}
	\begin{bmatrix}
	        A_1\exp(-x)\\
	        A_2\exp(x)\\
	        A_3\exp(6x)
	\end{bmatrix}\\
\end{aligned}\]

\[\boxed{\begin{cases}
		y_1=-5A_1e^{-x}+5A_3e^{6x}\\
		y_2=4A_1e^{-x}-3A_2e^x+4A_3e^{6x}\\
		y_3=3A_1e^{-x}+4A_2e^x+3A_3e^{6x}
\end{cases},\,A_1,A_2,A_3\in\mathbb R}\]


\section*2

\subsection*{(i)}
Let the characteristic polynomial of \(\vv A\) be \(p(x)\).

\[\begin{aligned}
	p(x)&=(x-\alpha)(x-\beta)(x-\gamma)\\
	    &=(x-\alpha)\left(x^2-(\beta+\gamma)x+\beta\gamma\right)\\
	    &=x^3-(\beta+\gamma)x^2+\beta\gamma x-\alpha x^2+(\alpha\beta+\gamma\alpha)x-\alpha\beta\gamma\\
	    &=x^3-(\alpha+\beta+\gamma)x^2+(\alpha\beta+\beta\gamma+\gamma\alpha)x-\alpha\beta\gamma
\end{aligned}\]

Given \(p(x)=x^3-x^2+kx+4\),
\[\begin{cases}
	\alpha+\beta+\gamma=1\\
	\alpha\beta+\beta\gamma+\gamma\alpha=k\\
	\alpha\beta\gamma=-4
\end{cases}\]

\[\vv A=\vv P\vv D\vv P^{-1},\,
	\vv P=\begin{bmatrix}\vv e_1&\vv e_2&\vv e_3\end{bmatrix},\,
	\vv D=\begin{bmatrix}\alpha&0&0\\0&\beta&0\\0&0&\gamma\end{bmatrix}\]

\[\begin{aligned}
	\vv B&=\vv P\begin{bmatrix}
		\alpha-\beta\gamma&0&0\\
		0&\beta-\gamma\alpha&0\\
		0&0&\gamma-\alpha\beta
	\end{bmatrix}\vv P^{-1}\\
	     &=\vv P\left(\begin{bmatrix}
		\alpha&0&0\\
		0&\beta&0\\
		0&0&\gamma
	\end{bmatrix}-\begin{bmatrix}
		\beta\gamma&0&0\\
		0&\gamma\alpha&0\\
		0&0&\alpha\beta
\end{bmatrix}\right)\vv P^{-1}\\
	     &=\vv P\begin{bmatrix}
		\alpha&0&0\\
		0&\beta&0\\
		0&0&\gamma
		\end{bmatrix}\vv P^{-1}-\vv P\begin{bmatrix}
		-4/\alpha&0&0\\
		0&-4/\beta&0\\
		0&0&-4/\gamma
\end{bmatrix}\vv P^{-1}\\
	     &=\vv P\vv D\vv P^{-1}+4\vv P\vv D^{-1}\vv P^{-1}\\
	     &=\boxed{\vv A+4\vv A^{-1}}
\end{aligned}\]

\subsection*{(ii)}
Consider the three eigenvalues of \(\vv B\):
\[\alpha-\beta\gamma=\alpha-\frac{-4}\alpha=\frac1\alpha\left(\alpha^2+4\right)\ne0,\,\forall\alpha\in\mathbb R\]
\[\beta-\gamma\alpha=\beta-\frac{-4}\beta=\frac1\beta\left(\beta^2+4\right)\ne0,\,\forall\beta\in\mathbb R\]
\[\gamma-\alpha\beta=\gamma-\frac{-4}\gamma=\frac1\gamma\left(\gamma^2+4\right)\ne0,\,\forall\gamma\in\mathbb R\]
Since \(\vv B\) has three non-zero eigenvalues, \(\det(\vv B)\ne0\implies\col(\vv B)=\mathbb R^3\).

Since the set of three eigenvectors is linearly independent:
\[\spn\{\vv e_1,\vv e_2,\vv e_3\}=\mathbb R^3\]

Therefore, because \(\spn\{\vv e_1,\vv e_2,\vv e_3\}=\col(\vv B)=\range(T)\) and \(|\{\vv e_1,\vv e_2,\vv e_3\}|=3=\dim\left(\mathbb R^3\right)\), \(\{\vv e_1,\vv e_2,\vv e_3\}\) forms a basis for the range of \(T\).\qed
\end{document}
