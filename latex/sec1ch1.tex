\documentclass[a4paper,12pt]{article}
\usepackage{amsmath}
\usepackage{amsfonts}
\usepackage{amsthm}
\usepackage{amsfonts}
\usepackage{amssymb}
\usepackage{setspace}
\doublespacing
\title{Factors and Multiples}
\date{}
\begin{document}
\maketitle
\section{The Integers}
The integers \(\mathbb Z\) are also called whole numbers.
\[\mathbb Z=\{0,\pm1,\pm2,\pm3,\dots\}\]

\section{Positive Integers}
The positive integers \(\mathbb Z^+\) are sometimes called the counting numbers or {\bf natural numbers}.
\[\mathbb Z^+=\{1,2,3,4,\dots\}\]
\subsection{Example}
List out the first 3 positive integers when they are sorted in an ascending order. How are they related to one another?
\[\boxed{1,2,3}\]
The subsequent number is always 1 higher than the previous number.

\section{Prime Numbers}
A positive integer is called a {\bf prime number} \(p\) if and only if \(p>1\) and {\bf cannot} be written as the product of two smaller natural numbers such that \(p=n_1n_2\quad (p>n_1,n_2\in\mathbb Z^+)\). The natural numbers greater than 1 that are not prime are called {\bf composite numbers}. The are {\bf infinitely many} prime numbers!
\[\mathbb P=\{2,3,5,7,\dots\}\]
\subsection{Example}
Write down the 5\(^{\text{th}}\) prime number when they are sorted in an ascending order.
\[\boxed{11}\]

\section{Prime Factorisation}
Prime factorisation allows a composite number to be rewritten into the product of its prime factors.
\[N=p_1^{n_1}\times p_2^{n_2}\times p_3^{n_3}\times p_4^{n_4}\times\cdots\]
\subsection{Exmaple}
Express 7650 as a product of its prime factors.
\[\begin{array}{r|l}
    2&7650\\
    5&3825\\
    5&765\\
    3&153\\
    3&51\\
    17&17\\
    &1
\end{array}\]
\[\begin{aligned}
    7650&=2\times5\times5\times3\times3\times17\\
    &=2\times3\times3\times5\times5\times17\\
    &=\boxed{2\times3^2\times5^2\times17}
\end{aligned}\]

\section{Highest Common Factor (HCF)}
The HCF of two or more numbers is the largest number
that divides all the numbers. When computing the HCF of two or more numbers, multiply the prime factors with the {\bf lowest powers}.
\subsection{Example}
Find the HCF of 12 and 18.
\[\begin{cases}
    12=2^2\times3\\
    18=2\times3^2
\end{cases}\]
\[\text{HCF}(12,18)=2\times3=\boxed{6}\]

\section{Lowest Common Mutiple (LCM)}
The LCM of two or more numbers is the smallest number
that is divisible by all the numbers. When computing the LCM of two or more numbers, multiply the prime factors with the {\bf highest powers}.
\subsection{Example}
Find the LCM of 12 and 18. Find the LCM of 12, 18 and 20.	
\[\begin{cases}
    12=2^2\times3\\
    18=2\times3^2\\
    20=2^2\times5
\end{cases}\]
\[\text{LCM}(12,18)=2^2\times3^2=\boxed{36}\]
\[\text{LCM}(12,18,20)=2^2\times3^2\times5=\boxed{180}\]

\section{Square Roots}
The square root of a number \(N\) returns a number \(n\) whose square is \(N\) (\(N=n^2\iff\sqrt N=n,\,n\geqslant0\)). Numbers whose square root is an integer are called {\bf perfect squares}.
\subsection{Example}
\subsubsection{Problems}
Express 676 as a product of its prime factors.\\
Express 676 in the form of \(n^2\).\\
Hence, find the square root of 676.\\
Explain whether 676 is a perfect square.\\
Is 16 a perfect square?
\subsubsection{Solutions}
\[676=\boxed{2^2\times13^2}\]
\[\begin{aligned}
    676&=2^2\times13^2\\
    &=(2\times2)\times(13\times13)\\
    &=2\times2\times13\times13\\
    &=2\times13\times2\times13\\
    &=(2\times13)\times(2\times13)\\
    &=(2\times13)^2\\
    &=\boxed{26^2}
\end{aligned}\]
\[\sqrt{676}=\sqrt{26^2}=\boxed{26}\]	
Since the square root of 676 is an integer, 676 is a perfect square.
\[\begin{aligned}
    16&=2^4\\
    &=2\times2\times2\times2\\
    &=(2\times2)\times(2\times2)\\
    &=(2\times2)^2\\
    &=4^2
\end{aligned}\]
Yes, 16 is a perfect square.

\section{Cube Roots}
The cube root of a number \(N\) returns a number \(n\) whose cube is \(N\)\\(\(N=n^3\iff\sqrt[3]N=n\)). Numbers whose cube root is an integer are called {\bf perfect cubes}. 
\subsection{Example}
\subsubsection{Problems}
Express 216 as a product of its prime factors.\\
Express 216 in the form of \(n^3\).\\
Hence, find the cube root of 216.\\
Explain whether 216 is a perfect cube.\\
Is 64 a perfect cube?
\subsubsection{Solutions}
\[216=\boxed{2^2\times13^2}\]
\[\begin{aligned}
    216&=2^3\times3^3\\
    &=(2\times2\times2)\times(3\times3\times3)\\
    &=2\times2\times2\times3\times3\times3\\
    &=2\times3\times2\times3\times2\times3\\
    &=(2\times3)\times(2\times3)\times(2\times3)\\
    &=(2\times3)^3\\
    &=\boxed{6^3}
\end{aligned}\]
\[\sqrt[3]{216}=\sqrt[3]{6^3}=\boxed{6}\]
Since the cube root of 216 is an integer, 216 is a perfect cube.
\[\begin{aligned}
    64&=2^6\\
    &=2\times2\times2\times2\times2\times2\\
    &=(2\times2)\times(2\times2)\times(2\times2)\\
    &=(2\times2)^3\\
    &=4^3
\end{aligned}\]
Yes, 64 is a perfect cube. 
\end{document}
