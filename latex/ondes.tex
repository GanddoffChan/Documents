\documentclass[a4paper]{article}
\usepackage{amsmath,amsfonts,amsthm,amssymb}

\begin{document}
"My aim has always been the liberation of sound; to throw open the whole world
of sound music." -- Edgard Varèse

In your opinion, how has Varèse liberated sound? Refer in you answer to
**at least three** works, only **one** of which may be a Focus Work. [30]

Three works: Amériques, Ionisation, Equatorial

In the past, composers often have to comply with the musical conventions of
their times. These conventions came in the form of tonality, harmonic functions
as well as form and structure. Although they ensure that the music produced is
musical and acceptable for the audiences at the time, following such rules
restricted access to all of the theoretically possible musical languages one
can imagine. Edgard Varèse is one of the first composers who realised the
disadvantage of following the rules and decided to liberate the sounds of music
in expense of breaking the rules. Moreover, Edgard Varèse explored new timbre
though percussion and electronic music instruments. That is probably what
he referred to as 'the whole world of sound music'. The way at which he had
achieved all these will be explained with three of his works: Amériques,
Ionisation and Equatorial, below.

Amériques is the first work which Varèse wrote after he arrived in America. As
a Frenchman, his aim for the work is to describe his first impressions of
Americas through the work. Having the ultimate goal of liberation of sound in
mind, Varèse started the piece with a solo alto flute playing a passage made of
dissonant intervals, such as minor seconds, major sevenths and augmented
fourths. Moreover, this unusual opening motif is followed by a disruptive
percussion and brass descending chromatic tutti loud call, which has no
correlation with the opening motif at all, showing no continuity between the
two gestures. This way of opening his work is therefore very progressive but at
the same time, very free (in terms of composition styles). This way of
introducing a piece of music threw the conventions out of the window as
traditionally a long-winded exposition section must be present to introduce the
main motifs and ideas. Furthermore, the work continues in this fashion of
sudden, unexpected and unprecedented development for a full twenty minutes or
so. The expression he had used in the work is truly free. Therefore, Varèse has
successfully reached the goal of liberating music.

In Ionisation, a work for only the percussion section, Varèse explored heavily
on the unique timbres of some special percussion instruments that were not
considered in the past. He broadened the definition of instruments from the
percussion section by including many day-to-day objects that most people see at
only making noise into his percussion line-up. For example, siren held long
tones loudly with varying dynamics and duration; crow calls are played in a
short motif consisting of three quavers in secession, perhaps resembling real
crows found in America; and metal sheets are played as the introduction,
suggesting the sounds of a thunderstorm. All these sounds expanded the library
of sounds possible to be used in a composition. This is also what it means to
throw open the whole world of sound music.

Equatorial is unique piece that heavily explores on electronic music and
vocals.  This work featured the Ondes Martinot and tenor vocals. The Ondes, the
first electronic instrument, was invented in 1924, right before equatorial was
written. The Ondes is very expressive, as it is capable of insanely huge
dynamics that ranges from very soft to so loud that it stands out from the
whole orchestra. Its refreshing timbre produced by electron tubes is very pure
and never heard of at that time. The work is opened by the Ondes, with high,
long and loud notes being played. The timbre of the Ondes is clearly pronounced
in the whole piece. This way, Varèse had explored the world of sound music as
he used timbre never used in the past.
\end{document}
