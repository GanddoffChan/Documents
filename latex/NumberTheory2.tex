\documentclass[a4paper]{article}
\usepackage{amsmath,amsfonts,amsthm,amssymb}
\usepackage{graphicx}
\usepackage{concmath}
\usepackage[T1]{fontenc}

\begin{document}
\section*1
\subsection*{(a)}
False. \[0+\sqrt2=\sqrt2\]
\subsection*{(b)}
Suppose \(x,y\in\mathbb Q\).
\[x+y=\frac ab+\frac cd=\frac{ad+bc}{bd}=\frac pq\in\mathbb Q\]
\[x\in\mathbb Q\land y\in\mathbb Q\implies x+y\in\mathbb Q\]
\[x+y\not\in\mathbb Q\implies x\not\in\mathbb Q\lor y\not\in\mathbb Q\qed\]

\section*2
\subsection*{(a)}
Suppose \(\sqrt3+\sqrt5\in\mathbb Q\).
\[\begin{aligned}
	\sqrt3+\sqrt5&=\frac pq\\
	q\sqrt3+q\sqrt5&=p\\
	q\sqrt5&=p-q\sqrt3\\
	5q^2&=p^2-2pq\sqrt3+3q^2\\
	2pq\sqrt3&=p^2-2q^2\\
	12p^2q^2&=p^4-4p^2q^2+4q^4\\
	16p^2q^2&=p^4+4q^4\\
	(4pq)^2&=\left(p^2\right)^2+\left(2q^2\right)^2\\
\end{aligned}\]
\(p\) must be even
\[\begin{aligned}
	\sqrt3+\sqrt5&=\frac pq\\
	q\sqrt3+q\sqrt5&=p\\
	q\sqrt3&=p-q\sqrt5\\
	3q^2&=p^2-2pq\sqrt5+5q^2\\
	2pq\sqrt5&=p^2+2q^2\\
	20p^2q^2&=p^4+4p^2q^2+4q^4\\
	16p^2q^2&=p^4+4q^4\\
	\left(p^2\right)^2+\left(2q^2\right)^2&=(4pq)^2\\
\end{aligned}\]
\[\begin{aligned}
	\sqrt3+\sqrt5&=\frac pq\\
	\sqrt3&=\frac pq-\sqrt5\\
	3&=\frac{p^2}{q^2}-2\sqrt5\frac pq+5\\
	-2q^2&=p^2-2q\sqrt5\\
\end{aligned}\]

\section*3
Suppose there exists two odd integers, \(m\) and \(n\),
such that the sum of their squares is divisible by 4.
\[\begin{aligned}
	m^2+n^2&=4k\\
	{(2a+1)}^2+{(2b+1)}^2&=4k\\
	4a^2+4a+1+4b^2+4b+1&=4k\\
	4a^2+4a+4b^2+4b+2&=4k\\
	2\left(a^2+a+b^2+b\right)+1&=2k\quad(\implies\impliedby)
\end{aligned}\]
The left hand side is odd and the right hand side is even, resulting in a contradiction.
Therefore, the sum of square of two odd integers is not dividible by 4.\qed

\section*4
Suppose their is sum odd interger \(a\) such that \(a^2+1=2^n\) with \(n\geqslant2\).
\[\begin{aligned}
	a^2+1&=2^n\\
	{(2k+1)}^2+1&=2^n\\
	4k^2+4k+1+1&=2^n\\
	4k^2+4k+2&=2^n\\
	2k^2+2k+1&=2^{n-1}\\
	2\left(k^2+k\right)+1&=2\left(2^{n-2}\right)\quad(\implies\impliedby)
\end{aligned}\]
The left hand side is odd and the right hand side is even, resulting in a contradiction.\\
Suppose their is sum even interger \(a\) such that \(a^2+1=2^n\) with \(n\geqslant2\).
\[\begin{aligned}
	a^2+1&=2^n\\
	{(2k)}^2+1&=2^n\\
	4k^2+1&=2^n\\
	2\left(2k^2\right)+1&=2\left(2^{n-1}\right)\quad(\implies\impliedby)
\end{aligned}\]
The left hand side is odd and the right hand side is even, resulting in a contradiction.
Therefore, there are no integers \(a\) and \(n\) with \(n\geqslant2\) and 
\(a^2+1=2^n\).\qed

\section*5
Suppose there is some positive integer \(c<\gcd(a,b)\) such that \(c=ax+by\)
for some integers \(x\) and \(y\).
\[\]

\section*6
Suppose \(n\) is a perfect square.
\[\begin{aligned}
	n&=m^2\\
	n&=\left(p_1^{k_1}p_2^{k_2}\cdots p_t^{k_t}\right)^2\\
	n&=p_1^{2k_1}p_2^{2k_2}\cdots p_t^{2k_t}\\
	n&=\left(p_1^{k_1}\right)^2\left(p_2^{k_2}\right)^2\cdots\left(p_t^{k_t}\right)^2\\
	\sqrt n&=\sqrt{\left(p_1^{k_1}\right)^2\left(p_2^{k_2}\right)^2
	\cdots\left(p_t^{k_t}\right)^2}\\
\end{aligned}\]
\end{document}
