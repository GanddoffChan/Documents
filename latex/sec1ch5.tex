\documentclass[a4paper,12pt]{article}
\usepackage{amsmath,amsfonts,amsthm,amssymb}
\usepackage{physics}
\usepackage{geometry}
\usepackage{booktabs}
\usepackage{setspace}
\doublespacing{}
\title{Algebraic Equations}
\date{}
\author{}
\begin{document}
\maketitle

\section{Linear Equations}
\[\begin{aligned}
    3(4x-1)&=7(2x-5)\\
    12x-3&=14x-35\\
    38&=2x\\
    x&=\boxed{16}
\end{aligned}\]

\section{Linear Equations with Fractions}
\[\begin{aligned}
    \frac{2p+1}6-\frac{6-5p}{5}&=\frac{12p-15}{10}\\
    5(2p+1)-6(6-5p)&=3(12p-15)\\
    10p+5-36+30p&=36p-45\\
    40p-31&=36p-45\\
    4p&=-14\\
    p&=\boxed{-\frac72}
\end{aligned}\]

\section{Word Problems}
\subsection{Example 1}
When a number is added to two-fifths of itself, the result is 35. What is the number?

Let \(x\) be the number.
\[\begin{aligned}
    x+\frac25x&=35\\
    5x+2x&=175\\
    7x&=175\\
    x&=\boxed{25}
\end{aligned}\]
\subsection{Example 2}
The sum of 3 consecutive even numbers is 42. Find the smallest number.

Let the smallest number be \(x\), so that the 3 consecutive even numbers are \(x\), \(x+2\) and \(x+4\).
\[\begin{aligned}
    x+(x+2)+(x+4)&=42\\
    x+x+2+x+4&=42\\
    3x+6&=42\\
    3x&=36\\
    x&=\boxed{12}
\end{aligned}\]
\end{document}
