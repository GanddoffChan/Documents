\documentclass[a4paper]{article}
\usepackage{amsmath,amsfonts,amsthm,amssymb}
\usepackage{CJKutf8}
\title{`comprehensive' comprehensive notes}
\author{$\mathbb{GONDRO}$です}

\begin{document}
\begin{CJK}{UTF8}{min}
\maketitle

\section{functions of quotes}
\begin{itemize}
	\item arouse the interests of the readers
	\item provoke readers' thought
	\item support the author's points of view
\end{itemize}

\section{functions of rhetorical questions}
\begin{itemize}
	\item lay emphasis to the point
	\item analogies
	\item helps readers better understand the author's ideas
	\item metaphors
	\item create a vivid idea or image in readers as they are familiar with the connotation of the word/phrase
	\item emotional statements
	\item signal to readers whether a statement is positive and being celebrated or negative and doubted or dismissed
	\item distance markers \(\{\text{they},\dots\}\)
	\item helps distance the author from a point of view they do not agree with
	\item intensifiers or qualifiers
	\item act as a condition for the thing they said to come true
\end{itemize}

\section{functions of punctuations}
\begin{itemize}
	\item inverted commas
		\begin{itemize}
			\item to convey sarcasm
			\item to question the validity of a statement
		\end{itemize}
	\item ellipses
		\begin{itemize}
			\item to show a repetitive cycle
			\item to create suspense
		\end{itemize}
	\item brackets
		\begin{itemize}
			\item to provide additional information not deemed as important
		\end{itemize}
	\item colon
		\begin{itemize}
			\item to connect related information
		\end{itemize}
	\item CAPITALISED, {\bf bold}, \emph{italicised} words
		\begin{itemize}
			\item to emphasise an idea or a point
		\end{itemize}
\end{itemize}

\section{questions on author's ability/purpose}
\begin{itemize}
	\item to reframe an issue in a way the readers can better understand
	\item to highlight the severity of an issue
	\item to evoke or arouse readers' emotions
	\item to rebut one's opponent
	\item to make a disclaimer
	\item to mock something
	\item to inform/advise the readers
	\item to express shock, disgust or disbelief
\end{itemize}

\section{to explain strategies used by the writer}
\begin{itemize}
	\item comparing similarities/differences between 2 things to prove a point
	\item comparing then and now to show change/tramsformation
	\item using analogy to prove a point
	\item using metaphor to contrast 2 objects
	\item use rhetorical questions to emphasise a point
	\item use stories to illustrate a point/theme
\end{itemize}

\section{to explain a simile}
\begin{itemize}
	\item (literal meaning of simile). likewise, (explain how simile links to context)
\end{itemize}

\section{to explain irony}
\begin{itemize}
	\item we expect ... however ...
\end{itemize}
		
\section{to explain tone}
\begin{itemize}
	\item positive tone: supportive/approving/optimistic
	\item negative tone: disapproving/mocking/dismissive/defiant
\end{itemize}
\end{CJK}
\end{document}
