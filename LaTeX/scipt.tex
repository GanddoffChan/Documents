\documentclass{article}
\usepackage{setspace}
\doublespacing
\begin{document}
Good morning everyone! My name is Gordon. This is Yutong, Chuan Yew, Kendrea, and Yitong.\par
 (5 s)\par
 I will be guiding you through the introduction and the analysis of case study. Yutong will be introducing the target situation. Chuan Yew, Kendrea and Yitong will elaborate on the proposed interventions. Welcome to Technology? So EZ!\par
 Have you ever wondered why elderly are often seen as digitally inferior? As Singapore progresses and endeavours to become a smart nation, technology advances at a very fast pace. Elderly currently face difficulties in keeping up with the advancement of technology. If this is left unresolved, they will have troubles integrating into society. \par
 In this project, we propose a 3-phase intervention. Phase 1 is a publicity campaign, phase 2 is the reality show, Technology? So ez! And phase 3 is the ‘Silver Techie’ carnival.\par
 The case study we have chosen to research on is workers disrupted by mental illness in the workplace. We gained meaningful insights from studying the pre-existing managements of the disruption caused by mental illnesses on patients in the workplace, guiding us in designing interventions to help alleviate the disruption elderly faces.\par
 Both the mental illness patients and elderly are uncomfortable with the social stigma towards them. For the mental illness patients, they are regarded as unfit to work due to their condition while the elderly are regarded as too old to learn new skills such as technology. These existing perceptions result in low participation of pre-existing managements.\par
 The following are the managements that we have studied.\par
 The first management is the counselling service provided by the Employee Assistance Programme. Employees can choose between face-to-face  or video counselling. For offline counselling, employees risk being identified as a mental health patient, causing reluctance to go for counselling services, rendering measures to be ineffective. On the other hand, online counselling is convenient and accessible  for patients seeking help  as they do not have to be there physically. \par
 The second management is roadshows brought onsite. Roadshows are held at the sites of companies. For example, the 4-hour long stress management roadshow by the Health Promotion Board allows employees to visit the roadshow at their own convenience. Staff will be equipped with effective stress management skills through interactive activities. For example, terrarium-making is used to involve the workers. This way, mental illness patients are more engaged and willing to participate when they are preoccupied with fun-filled hands-on activities.\par 
 We have derived three lessons learnt from these two managements. Firstly, we have learnt that interventions should be designed such that they are convenient and accessible  to the public, attracting greater participation amongst our target audience. Subsequently, by utilising interactive  methods such as hands-on activities, the target audience would be more enticed to participate. Lastly, interventions that focus on spreading awareness  of the target situation should be designed, eliminating stigma. In summary, I have introduced our project and given an analysis of our case study, deriving the lessons learnt. Now I will hand the time to Yutong, who is going to introduce the target situation.\par
\end{document}
