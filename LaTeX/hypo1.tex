\documentclass[a4paper]{article}
\usepackage{amsmath,amsfonts,amsthm,amssymb}
\newcommand{\p}{p\text{-value}}

\begin{document}
\section*1

Let the mean mass of the contents of a bag be \(\mu\) grams.
\[\begin{cases}
	H_0:\mu=10000\\
	H_a:\mu\ne10000
\end{cases}\]

\[\bar x=\frac{\sum(x-10000)}{80}+10000\approx9968.6\]
\[s^2=\frac1{79}\left(\sum(x-10000)^2-\frac{\left(\sum(x-10000)\right)^2}{80}\right)\approx24448\]

Under \(H_0\), the test statistic \(\bar X\sim N\left(\mu,\frac{s^2}{n}\right)\) approximately.
\[\p=2P\left(\bar X\leqslant\bar x\right)\approx0.072463\leqslant0.10\]

Since \(\p\leqslant10\%\), \(H_0\) is rejected in favour of \(H_a\) and a conclusion is made that there is sufficient evidence at a 10\% significance level that the mean mass of a bag differs from 10 kg.

`at the 10 \% significance level' means that there is a 10\% probability when the null hypothesis of the mean mass of the contents of a bag does not differ from 10 kg is rejected when in fact the mean mass is 10 kg.

`\(\p\)' is the probability at which the test statistic \(\bar X\) is more or equally extreme as the sample statistic \(\bar x\).

\newpage

\section*2
Let the children in Ms Patricia's school sleep an average of \(\mu\) hours.
\[\begin{cases}
	H_0:\mu=\mu_0=6.5\\
	H_a:\mu<\mu_0=6.5
\end{cases}\]

\[\bar x=\frac{\sum x}n=6.325\]
\[s^2=\frac n{n-1}\left(\frac{\sum x^2}n-\bar x^2\right)\approx0.12214\]

Under \(H_0\), the test statistic \(T=\frac{\bar X-\mu_0}{S\sqrt{\frac1n}}\sim t(n-1)\).
\[\p=P\left(T\leqslant\frac{\bar x-\mu_0}{s\sqrt{\frac1n}}\right)\approx0.099807>0.08\]

Since \(\p>8\%\), \(H_0\) is not rejected and a conclusion is made that there is insufficient evidence at a 8\% significance level that the children in Ms Patricia's school sleep an average of fewer than 6.5 hours each night.
\\
\[T\sim t(14)\]
\[s^2=\frac{15}{14}(0.849)=\frac{2547}{2800}\]
\\
\[\begin{aligned}
	\p&\leqslant0.08\\
	P\left(T\leqslant\frac{\bar x-6.5}{\frac{2547}{2800}\sqrt{\frac1{15}}}\right)&\leqslant0.08\\
	\frac{\bar x-6.5}{\frac{2547}{2800}\sqrt{\frac1{15}}}&\leqslant a\approx-1.4839\\
	\bar x&\leqslant\frac{2547a}{2800\sqrt{15}}+6.5\approx6.1515\\
\end{aligned}\]
\[\bar x\in[0,6.15]\]
\end{document}
