\documentclass[a4paper]{article}
\usepackage{amsmath,amsfonts,amsthm,amssymb}
\usepackage{graphicx}
\usepackage{concmath}
\usepackage[T1]{fontenc}
\usepackage{geometry}
\title{William Schuman American Festival Overture}
\author{Gordon Chan}

\begin{document}
\maketitle

Schuman developed his melodies. There is a three-note motive for the opening of the piece. The motive is very distinctive and recognisable. The motive definitely entails festive emotions, where people come together a festive occasion of some sort. This very naturally suggested that the work itself is a piece of music being composed for a very festive occasion. The development of this bit of 'folk material', then, is along purely musical lines.\par
The first section of the work is concerned with the opening material and the ideas growing out of it. This music leads to a transition section and the subsequent announcement by the violas of a Fugue subject. The entire middle section is given over to this Fugue.  This climax leads to the final section of the work, which consists of opening materials recapitulated and the introduction of new subsidiary ideas.\par
In terms of the rhythms used, the metre used is \(\begin{array}{c}4\\4\end{array}\). The tempo of the work is rather fast. There are many triplet ideas, from quaver triplets to crochet triplets. Such triplets gives a poly rhythmic feel to the work in the opening. The cooperated syncopation makes the rhythm of the work non-trivial and hence interesting.\par
The approach to rhythm is quite different to the focus work Ameriques and 3PINE. Both of the focus works feature intensive use of metre changes to create rhythmic ambiguity. However, in terms of the use of poly rhythm, American Festival Overture can still be considered as similar to the focus works, where poly rhythms are heavily implemented.\par
Considering the harmonic aspect of the work, there are many intervals of thirds and fourths. The intervals are present both in the melodies and the chordal harmony.\par
This is also very different from the focus works. Ameriques emphasises on unconventional, dissonant intervals such as seconds, sevenths and ninths as seen from the opening solo alto flute melody. 3PINE puts many different themes that are in different keys in the same time to create some sort of poly tonal strategies. Both approaches to harmony are distinct from the more tonal, consonant approach of American Festival Overture.\par
William Schuman utilised the orchestral forces to further reinforce to festive picture that the work is trying to portray. The orchestration is at first for strings alone, later for woodwinds alone and finally, as the Fugue is brought to fruition, by the strings and woodwinds in combination. Generally, the orchestration is strong and huge, typical of modern orchestras. The use of orchestral forces in the work is similar to the focus works, where large, forceful sounds are also produced by enormous orchestras.

\end{document}
