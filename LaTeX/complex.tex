\documentclass[12pt,a4paper]{article}
\usepackage{amsmath}
\usepackage{amsfonts}
\usepackage{amsthm}
\usepackage{amsfonts}
\begin{document}
\section*{(a)}
\[2z^3-3z^2+kz+26=0,\,k\in\mathbb R\]
By the conjugate root theorem, since all the conefficient of \(z\) are real, if \(z=1+ai\) is a root to the equation, \(z=1-ai\) must also be a root to the equation. Also, any odd degree polynomial has at least one real root. Thus, the original expression in \(z\) can be factor as:
\[\begin{aligned}
	&\phantom=A(z-(1+ai))(z-(1-ai))(z-c)\\
	&=A((z-1)-ai)((z-1)+ai)(z-c)\\
	&=A\left((z-1)^2-(ai)^2\right)(z-c)\\
	&=A\left(z^2-2z+1+a^2\right)(z-c)\\
	&=Az^3-2Az^2+A\left(1+a^2\right)z-Acz^2+2Acz-A\left(1+a^2\right)c\\
	&=Az^3-A(2+c)z^2+A\left(1+a^2+2c\right)z-A\left(1+a^2\right)c\\
\end{aligned}\]
Since this expression is identical to the original,
\[Az^3-A(2+c)z^2+A\left(1+a^2+2c\right)z-A\left(1+a^2\right)c\equiv2z^3-3z^2+kz+26\]
By comparing coefficients, 
\[\begin{cases}
	A&=2\\
	A(2+c)&=3\\
	A\left(1+a^2+2c\right)&=k\\
	A\left(1+a^2\right)c&=-26
\end{cases}
\implies
\begin{cases}
    2(2+c)&=3\\
    2\left(1+a^2+2c\right)&=k\\
    2\left(1+a^2\right)c&=-26
\end{cases}\]
\[\begin{cases}
	c&=-\frac12\\
    2\left(1+a^2+2c\right)&=k\\
	2\left(1+a^2\right)c&=-26
\end{cases}
\implies
\begin{cases}
	2\left(1+a^2+2\left(-\frac12\right)\right)&=k\\
	2\left(1+a^2\right)\left(-\frac12\right)&=-26
\end{cases}\]
\[\begin{cases}
	2a^2&=k\\
	a^2&=25
\end{cases}
\implies
\boxed{\begin{cases}
    k&=50\\
    a&=5
\end{cases}}\]
\newpage
\section*{(b)}
\subsection*{(i)}
\[\begin{aligned}
    (x+iy)^2&=15+8i\\
	x^2+2ixy-y^2&=15+8i\\
	\left(x^2-y^2\right)+(2xy)i&=15+8i
\end{aligned}\]
Since \(x,y\in\mathbb R\), by comparing the coefficients of the real and imaginary parts,
\[\begin{cases}
	x^2-y^2&=15\\
	2xy&=8
\end{cases}\]
\[\begin{aligned}
	y=\frac4x\implies x^2-\left(\frac4x\right)^2&=15\\
	x^2-\frac{16}{x^2}&=15\\
	x^4-16&=15x^2\\
	x^4-15x^2-16&=0\\
	\left(x^2+1\right)\left(x^2-16\right)&=0\\
	x^2-16&=0\\
	x^2&=16\\
	x&=\pm4\\
	y&=\frac4{\pm4}\\
	y&=\pm1
\end{aligned}\]
Possible values: \((x,y)=\boxed{(-4,-1)}\text{ OR }\boxed{(4,1)}\).
\subsection*{(ii)}
\[\begin{aligned}
    z^2-(2+7i)z&=15-5i\\
    z^2-(2+7i)z-(15-5i)&=0\\
    z&=\frac{-(-(2+7i))\pm\sqrt{(-(2+7i))^2-4(1)(-(15-5i))}}{2(1)}\\
    &=\frac{(2+7i)\pm\sqrt{(2+7i)^2+4(15-5i)}}{2}\\
    &=\frac12\left(2+7i\pm\sqrt{4+28i-49+60-20i}\right)\\
    &=\frac12\left(2+7i\pm\sqrt{15+8i}\right)\\
    &=\frac12\left(2+7i\pm\sqrt{4^2+2(4)(i)+i^2}\right)\\
    &=\frac12\left(2+7i\pm\sqrt{(4+i)^2}\right)\\
    &=\frac12(2+7i\pm(4+i))\\
\end{aligned}\]
\[\begin{aligned}
    z_2&=\frac12(2+7i-(4+i))\\
    &=\frac12(-2+6i)\\
    &=-1+3i\\
    &=\sqrt{(-1)^2+3^2}e^{i\left(\pi-\tan^{-1}3\right)}\\
    &=\boxed{\sqrt{10}e^{i\left(\pi-\tan^{-1}3\right)}}
\end{aligned}\]
\\
\[\begin{aligned}
    z_1&=\frac12(2+7i)+(4+i))\\
    &=\frac12(6+8i)\\
    &=3+4i
\end{aligned}\]
\\
\[\begin{aligned}
    \arg z_1^2z_2^*&=2\arg z_1-\arg z_2\\
    &=2\tan^{-1}\frac43-\left(\pi-\tan^{-1}3\right)\\
    &=\boxed{2\tan^{-1}\frac43+\tan^{-1}3-\pi}
\end{aligned}\]
\end{document}
