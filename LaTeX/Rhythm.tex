\documentclass[11pt]{article}
\usepackage[margin=1in]{geometry}
\title{Write-up on \textbf{Wonky Steps} (draft 3)\\
\Large(composition technique: rhythmic counterpoint)}
\author{}
\date

\begin{document}

\maketitle

\section{Compositional Approach}
The piece, \textbf{Wonky Steps}, explores the use of rhythmic counterpoint to create a sense of unsteadiness in the music. Overall, approaches of this piece include metric displacements, poly-rhythms and canons.
\section{Structure and Motive Development}
The piece opens with a series of falling fifth motif in both the piano and double bass. After that, the double bass plays a bass line based on what the clarinet plays for the upper melody. This rhythm of the clarinet melody is constantly being imitated by the piano throughout bar 5 to 10 by repeating the same rhythm a few quavers after the clarinet plays the same rhythm. This is resulted from the metric displacement approach. A bar of transition is used to change the rhythmic character at bar 12, to link to a new section. The upper melody in bar 12 played by the clarinet is first played by the piano at bar 11 to make the listeners anticipate the change in rhythmic qualities before the actual change happens.\\

The main theme in the second section is taken by the double, in a poly-rhythmic fashion against the piano part. In the second section, the double bass plays a phrase of repeating rhythm which has the exact same rhythm as the phrase played in unison at the first bar of the section. On the other hand, the piano the repeating another seven-beat rhythm that emphasises all the even quavers of a bar while right hand plays three quavers in unison with the double bass at the end of each bar. The three quavers at the end emphasises the connect between the piano and double bass parts despite the difference in their rhythm. The clarinet then plays an rising scale of a pattern of whole-tone, whole-tone, whole-tone and semitone from G4 all the way up to E6. It then plays a falling scale of a pattern of whole-tone, whole-tone and semitone from E6 all the way down to F\(\sharp\)3, landing on the third section.\\

The third section spans from bar 17 to bar 20. In this section, the double bass repeats yet another rhythm while outlining the chordal qualities of the bars. The main effect create in this section however comes from the piano part. The piano plays a ever increasingly intense rhythm as the music progresses. In particular, in the first bar of the section, the rhythm consists of two crochets and two dotted quavers. In the second bar, it changes to three triplets and two dotted quavers. This pattern continues until on the fourth bar, there are five quintuplets and two dotted quavers. As a result, a sense of increasing excitement is created as the space between each note decreasing as the music progress.\\

The forth section begins on bar 21 and ends on bar 24. In this section, the three instruments are playing in unison to an irregular rhythmic to resolve to an F\(\sharp\) major chord. Every note other than the last three notes in this section are accented so that the rhythmic qualities of the section can be projected clearly. The rhythm used here consists of 2 groups of 4 quadruplets and 2 groups of 4 regular quavers. It is also worthy to note that the these groups are arranged in such a way that the rhythm produced can be classified as non-retro-gradable.\\

The fifth and final section starts from bar 25 till the end of the piece. Canon is used in this section. It consists of a series of rising fifth motif with the rhythm that can be derived from retrograding the opening motif.  The canon resolves to a C\(\sharp^{(add\,2)\,omit\,5}\).
\section{Musical Effects}
The unsteady effect is mainly achieved through the use of irregular metre and poly-rhythms, which blurs the metered character and leads to uneven feeling, which result in a sense of uncertainty.
\section{Technique Explored}
The rhythmic counterpoint is utilised throughout the piece. For example, from bar 3 to bar 11 as the double bass plays a two-bar rhythmic cycle. This cycle repeats regardless of what is being played on top of the bass line by the piano and clarinet. This creates a rhythmic counterpoint between the bass line and the upper voices. A similar approach is used from bar 13 to 20, where the bass plays repeated poly-rhythmic motives while piano and clarinet playing other higher voices. It is important to also note that the bass line is supporting the higher voices even when playing some wonky rhythms.\\

Other techniques are also implemented to achieve a rhythmic counterpoint, such as a metric displacement in the piano's line from bar 5 to bar 8 where semiquaver rests are constantly being added each time the line repeated, and a  non-retro-gradable rhythm from bar 21 to 22.
\end{document}
