\documentclass[a4paper]{article}
\usepackage{amsmath,amsfonts,amsthm,amssymb}
\title{Complex Numbers}
\date{\(\mathbb{THEODORE}\)}
\author{\LaTeX}
\begin{document}
\maketitle

\section{Definition of the Imaginary Unit}

\[i^2=-1\]

\section{Rectangular Form}

\[z=x+iy,\,x,y\in\mathbb R\]

\subsection{Conjugate}

\[z:=x+iy\iff z^*=x-iy\]

\section{Polar Form}

\[z=re^{i\theta},\,r\in[0,\infty),\,\theta\in(-\pi,\pi]\]

\subsection{Conjugate}

\[z:=re^{i\theta}\iff z^*=re^{-i\theta}\]

\section{Equation Relating the Two Forms}

\[r=\sqrt{x^2+y^2}\]

\[\theta=\arg(x+iy)=\begin{cases}
    -\pi+\tan^{-1}\frac yx,\,x<0,y<0\\
    -\frac\pi2,\,x=0,y<0\\
    \tan^{-1}\frac yx,\,x>0,y<0\\
    \tan^{-1}\frac yx,\,x>0,y\geqslant0\\
    \frac\pi2,\,x=0,y>0\\
    \pi+\tan^{-1}\frac yx,\,x<0,y\geqslant0\\
\end{cases}\]

\section{Conjugate Root Theorem}

If \(p(z)\) is a polynomial in \(z\) whose coefficients are only real and \(p(z)=0\) has root \(z_1\), then \(z_1^*\) is also a root.

\end{document}
